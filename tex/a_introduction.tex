
\label{section:Introduction}

\section{The continuous growth of the air transportation industry}%
\label{grow}

The air transportation industry has been growing since 1945. It passed through several economic and geopolitical crises, the most recent being the COVID-19 pandemic. The latter has significantly impacted the world passenger traffic, reducing the industry by 40\%, when comparing 2021 against 2019, before the spread of the virus. This diminish of the sector cost approximately 324 billion dollars of gross passenger operating revenue for airlines \cite{ICAOEconomicImpact} The COVID-19 also influenced the airport's revenue and operations. At the beginning of the pandemic, the close sky policies allowed only essential flights, reducing the use of the airport's infrastructure. Sanitary measures also changed the aircraft movement itinerary, due to mandatory cleaning and inspections before and after transporting passengers, which directly affected the demand that could be accepted by the airports. Even through all these negative indicators, it is estimated a recovery of almost 103\% of worldwide passenger transportation in 2024, when comparing against 2019 \cite{IATA2024}. Some regions, such as Central America, have projections that surpass the passenger demand of 2019 at the end of 2023. Therefore, even with this uncertainty, the increasing demand required by the airlines, to compose a better network, will result in congestion of several airports, thereat old discussions of how to efficiently resolve the distribution of aircraft operations at airports come back into evidence.

\section{Airport Conception}
\label{concep}

In order to better understand this study's main problem, it is important to have knowledge of the basic concepts of airport’s operation. The prototypical airport is composed of 2 parts: an (i) airside and (ii) landside \cite{Lance2009}. From an operational airport’s perspective, the aircraft movement starts at the airside, when it approaches the runway through the commands of the Air Traffic Management (ATM). After the landing at the runway, the aircraft continues the movement, now at the ground level, through the taxiways. The aircraft approaches the apron in the direction of the parking stand. When it is positioned at its designated stand, the aircraft parks and bridges or airstairs are attached to the equipment in order to disembark the passengers, leading them to a gate. After the gate, the passengers are at the landside, in the airport terminal. There, they may enjoy the commercial stores (e.g., convenience stores, restaurants, and car rental agencies) before collecting their luggage and leaving the airport by ground transport or connecting to other flights. 

The opposite movement is also possible: the passengers arrive at the airport using ground transportation, they may also use the airport’s parking lots; they access the terminal, after check-in procedures and security inspections; they may enjoy the commercial stores, before embarking to the aircraft; at the airside, the aircraft pushes back, after it gets disattached from the airport bridges or airstairs; it proceeds through the taxiways until it arrives at the runways; and finally it takes-off, ending its movement at the airport. 

With the prototypical airport conception, it is possible to start the discussion about two main concepts: (i) how the airport generates revenue, and (ii) what are the main limitations of the airport infrastructure. These two are main components of the airport’s slots allocation problem.

\section{Airport Revenue Generation}
\label{revGen}

The airport revenue is mainly generated through aeronautical and non-aeronautical sources, both of which are directly related to the aircraft traffic volume. The former is composed by: the fare paid by each passenger using the airport infrastructure, usually included in the flight cost paid to the airlines; and the movement charge, which is paid by the airline when it lands, uses the infrastructure (e.g., hangar and terminal), and takes-off from the airport. The latter is the revenue generated by: the commercial concession at the terminal, such as rental fees for the retail at the airport area; the real estate exploitation, which includes hotels, restaurants, and parking lots at the airport perimeter; and services offered to passengers and airlines. A market equilibrium between the two revenues sources is needed, since reduction of the aeronautical charges commonly means stimulation to the non-aeronautical revenues, through passenger demand increments, but it does not guarantee the a overall maximum return \cite{ICAO2013}.

In Europe (2019), it is shown by \citeonline{ACI2021} that 54\% of the overall airport revenue was generated by aeronautical sources. \citeonline{ICAO2013} points out that 63\% of the  aeronautical income comes from charges directly applied to passengers. The article also argue that airport users do not pay the full cost of the infrastructure the use, explaining that, in 2012, 69\% of the airports worldwide were loss-making. Seeking to profitability, the airports have increased the variety of commercial activities. It can cushioning the impact of lower passenger and freight volumes, due to its higher profit margins. This creates a cross-subsidization between aeronautical and non-aeronautical revenues sources, since profits from commercial exploitation, for example, can be reinvested in airport infrastructure, increase the demand of passenger or aircraft that can be accepted at the airport, reducing capital needs and overall cost.

Knowing that airport are important for a country social-economic-growth and, as it was said, with an efficient management it can achieve profitability, governments have decided that, under the right economic conditions, they can included private sector participation (e.g., outright ownership, short or long-term concessions, public-private-partnership schemes, and management contracts) for the financing and operation of an airport infrastructure. The airports, under this new ownership model, have sought for better ways to compete for both passenger and airlines. The aviation community agrees with the need of infrastructure investments to accommodate the growth of the industry, increasing the profit margins. Since most of the infrastructure added are in large increments, and the time needed for these ventures is extensive, the airports are exposed to considerable risk \cite{ICAO2013}. Even with the constant pressure for the airport expansion, most of the administrator have to better manage its current infrastructure, until the investment risks are reduced to an acceptable level.

\section{Airport's infrastructure capacity}
\label{infra}

The limitation of the airport’s infrastructure use is related with \cite{ARC2021, WASG2020}:
\begin{enumerate}
	\item How many aircraft movements the airport’s runways can accept in a certain period (e.g., one hour, 30 minutes, 15 minutes, and 5 minutes);
	\item How many stands are available at the airport, and what are their category (e.g., cargo or passenger commercial flights);
	\item How many gates are available for the passenger movements; 
	\item How many passengers can be processed through the security inspection at the check-in;
	\item How many passengers are acceptable for the terminal area available; and
        \item How many aircraft can use the infrastructure through time slot and environmental restrictions.
\end{enumerate}%

The most restrictive infrastructure capacity can be used to accept movements at the airport, or a combination of several limitations can be best suited for the task to distribute slots without damaging the airport’s level of service, such as movement delays and congestion at the taxiways. 


\section{Problem statement}
\label{probl}

The present study introduces two main reasons to elevate the efficiency in the initial slot allocation in an airport. The first reason is related to the airport's ways to generate revenue. The two main ways of an airport generates revenue (i.e., aeronautical and non-aeronautical) are directly related to the passenger traffic volume. In \cite{graham2009important}, for example, it is discussed the importance of the commercial revenues, which represented (2017) 52.9\% of the total airport revenue to Africa and the Middle East, 45.7\% to Asia and Pacific, 38.1\% to Europe, 52.6\% to North America, and only 29.0\% to Latin America and the Caribbean. The article blames the small exploitation of airports' commercial areas in Latin America and the Caribbean on the lack of airport traffic stimulus, which results in the full dependence of the regulated airport fares, and non-development in both commercial and aeronautic areas.
In the Economic Commission for America Latin and the Caribbean \cite{planzer2019airport}, it is demonstrated the growth of the worldwide air transportation market, before the COVID-19 pandemic (2017), which, by International Civil Aviation Organization (ICAO), would reach 10 billion passengers per year by 2040. Despite the great projection, the report also describes the Latin American and the Caribbean challenges to build airport infrastructure, the major impediment to the development of the local sector. The International Air Transport Association (IATA) has urged the local aviation authorities to ensure, between several highlighted topics, the efficient use of the available airport’s capacity in the region. When the expansion of infrastructure it is no possible, the optimization of the allocation in the available slots is strongly sought by the airport administrator, to improve the revenue generation prospects. Some regions have already implemented more restrictive on-time performance goals for airlines \cite{ravizza2014aircraft}, which aims to enhance aircraft's taxi times, take-off sequencing, and allocating push-back time, thus affecting the demand of movements that an airport can accept.

The second reason is related to the impact of the operational performance improvement in the airlines. As it was mentioned, the operational capacity of an airport, i.e., the ability to absorb the demand of flights, is directly related to its infrastructure (e.g., number of runways, gates, aircraft stands, and terminal areas). Although the mentioned features of an airport can be used to determine its capacity, a lower limit is set to mitigate the negative impacts of the airport’s main partners, the airlines. This measure goes in the opposite direction of the first reason. When the limit is set greater than it should be, operational impacts are raised (e.g., delays). Most airlines interconnected their flights in a network. Therefore, a delay, for example, does not only impact the airline locally, but it can also spread its repercussions through other airports, impacting several aircraft itineraries and elevating the airline’s operational costs. This topic is discussed in \cite{ball2010total}, where it is described that the airline's costs of crew management, disrupted passenger accommodation and aircraft re-position, due to delays, can reach 20 billion dollars a year (2017). Another negative impact is related to intense and not planned aircraft taxi traffic. In \cite{nikoleris2011detailed}, it is reported that the fuel consumed in the stop-and-go situations, resulting primarily from the congestion on the airport’s taxiways system, can be approximately 18\% of the fuel used during the whole operation.

For that reason, while both airports and air operators want to expand their operations, a balance between accepted demand, which increases revenues (for airports), and operational efficiency, which mitigates high operational costs (for airlines), is needed.

\section{Known approaches}
\label{appr}

As it is highlighted in \cite{cavusoglu2021minimum} , there are three predominant processes for the airports’ capacity allocation: (i) administrative management, with slot distribution processes through a set of specific prioritization rules; (ii) economic management, where the airport’s congestion price is established, slot sales and exchanges are allowed, or auctions are made; and (iii) no regulatory mechanism, relying only on some infrastructure constraints as parameters (e.g., LaGuardia Airport and Ronald Reagan Washington National Airport as presented by \cite{FAAslots}).

The administrative management is the most used mechanism for slots. It is often use the Worldwide Airport Slot Guidelines (WASG), published by (IATA) \cite{ribeiro2018optimization}. There are also applications of WASG variations through local resolutions, as is the case of Brazil, where the local regulator is the National Civil Aviation Agency of Brazil (ANAC). Therefore, this chapter will: (1) present administrative mechanisms, through the WASG and its allocation practices in congested airports, discussing the differences and similarities between the resolution covering Brazilian airports and the best practices developed by IATA; (2) briefly explain the characteristics of the economic management, which contains market-driven mechanisms; and (3) review advantages of slot management mechanism applications compared to airports with almost no restrictions.

\section{Administrative demand management}
Demand management has been demonstrated in several works as one of the main processes to distribute air operators’ movements in the airports’ available infrastructure, dealing with the lack of capacity, delays and operational service level decaying due to congestion \cite{zografos2017increasing, ribeiro2018optimization}.

The administrative demand management approach aims to increase airport efficiency with allocation prioritization. Based upon optimization techniques, such practices have as objective to allocate in a certain number of available resources (airport slots), the dependent activities, i.e., air operators’ arrivals and departures movements. The allocation, made through the distribution of slots, must satisfy specific conditions (explained in \ref{infra}) \cite{zografos2017increasing}.

At the strategic level, the administrative management develops a cooperation practice where actions are coordinated between stakeholders, with bilateral communications and frequent reviews, so that the allocation may take place in accordance with all parties. The general objective is to match the requests made by the air operators with the declared infrastructure capacity, maximizing the number of accepted movements and, at the same time, minimizing the air operators’ operational cost to when they need to reschedule its flights outside the planned hours, due to the restrictions imposed at the airport \cite {gillen2016airport}. As it was explained, the main administrative management strategy is the IATA’s WASG, which has been adopted by local authorities in several countries \cite{zografos2017increasing, fairbrother2018development, cavusoglu2021minimum}.

\subsection{Worldwide Airport Slot GuideIines (WASG) - IATA}
To provide a standard for slot management and operations planning, the IATA, together with Airports Council International (ACI) and Worldwide Airport Coordinators Group (WWACG), publishes the WASG \cite{WASG2020}.

In general terms, the proposal to coordinate airports through the WASG aims to: (i) improve the connectivity of air services, promoting competitiveness in congested airports; (ii) provide slot schedules that meet demand and are consistent across seasons; (iii) ensure a fair, transparent and non-discriminatory slots allocation; (vi) promote efficient use of airports’ infrastructure; (v) guarantee market access to new air operators; (vi) provide a flexible slot management system for the industry, which can be adapt through local regulations and market conditions; and (vii) minimize congestion and delays caused by the allocation of slots with no parameters.

\subsubsection{Slot Stakeholders}

Slot management impacts several groups that are interested in the growing demand and the efficiency of the airport infrastructure use. Among them, the WASG highlights: 

\begin{enumerate}
	\item The air operators, the origin of the movement demand in the airports;
	\item The airport administrator, responsible for balancing the number of slots that will be made available and the adequate operational level;
	\item The authorities responsible for airspace control; 
	\item Coordinators or facilitators responsible for managing slots at the airport; and
        \item The government responsible for the airport use regulation.
\end{enumerate}%

The \refFig{fig_stakeholders_general} illustrates the relationship of stakeholders, showing the information that is exchanged during the slot management process.

\figuraBib{1_stakeholders_en}{Relationship between stakeholders in the management of slots through the direct use of WASG}{}{fig_stakeholders_general}{width=1\textwidth}%

Air operators have the responsibility to send requests for desired slots through a message, detailing the operations, and to prepare to cases where congestion is present in the airport and offers with different slots are sent as a response to the request. They also must follow the calendar of coordination activities and local regulations, monitoring the slots allocated to always mitigate the risk of airport infrastructure misuse.

The coordinator or facilitator are responsible to allocate the air operators slots in a transparent and non-discriminatory way. They must ensure that the allocation is made in accordance with the capacity declared by the airport and the priority criteria established in the WASG. During the activities, the they assist the understanding of the coordination parameters, local guidelines, regulations, and any other criteria used in the allocation. They must make available the list of the slots allocated and the offers sent, explaining why the requests were not fulfilled. They must attend the conferences, in order to better approach the needs of the air operators, trying to find a middle ground between competitors. They also must perform slot monitoring, to mitigate infrastructure misuse and to apply the use-or-lose-it rules, creating the historical slots.

The airport administrator must provide support to the facilitator or coordinator, assisting in their tasks. Airports must provide the infrastructure information so that capacity is filled with adequate demand. If there is any special limitation present in their capacity, airports must communicate with the other stakeholders, especially if the demand projection shows to exceed the available capacity in future seasons. Capacity updates must be presented at least twice a year, following each season's activity calendar. The airport must send the monitoring of the slots to the airspace authorities and local regulators.

The authorities responsible for air control must communicate to the airport administrator the airspace capacity that modifies the airport infrastructure capacity parameters. The government that concedes the airport, must, through regulations, send the airport operator the conditions for using the infrastructure, as well as inform the investments that the airport operator needs to do in order to meet the projected target demand.

\subsubsection{Capability Analysis}

Prior to the slot allocation, a demand and capacity analysis is performed. This analysis must be carried out promptly for the declaration of capacity, one of the season's official activities. It takes into account the airport's ability to absorb movement demand, together with airspace limitations, in a level that the quantity of movements accepted in the airport infrastructure does not degrade its service levels, i.e.,  does not promote congestion and delays. The result is constraints regarding the runway(s), stand(s), terminal(s), gate(s), among many other relevant airport infrastructures.

Another important limitation of capacity is environmental, i.e., the airport discusses the need to reduce the number of slots available when there is a relevant environmental impact (e.g., periods without movement due to noise pollution or due to environmental events that affects the aircrafts’ visual operations). 

One of the objectives of the capability statement is to understand which level the airport is at, i.e., Level 1, Level 2, and Level 3. The level indicates the the airport's coordination criticality. 

In order to better understand the the slot management process, \refTab{tab:def} shows a glossary containing the main terms used by WASG.

\begin{center}
    
\setlength{\tabcolsep}{10pt} % Default value: 6pt
\renewcommand{\arraystretch}{1.5} % Default value: 1
\begin{xltabular}{\textwidth}{p{0.5cm} p{4cm} p{10cm}}
\caption{Glossary of relevant terms in WASG's slot management \cite{WASG2020}.} \label{tab:def} \\

\hline \multicolumn{1}{l}{\textbf{\#}} & \multicolumn{1}{l}{\textbf{Term}} & \multicolumn{1}{l}{\textbf{Definition}} \\ \hline 
\endfirsthead

\multicolumn{3}{c}%
{\tablename\ \thetable{} - Continued from previous page.} \\
\hline \multicolumn{1}{l}{\textbf{\#}} & \multicolumn{1}{l}{\textbf{Term}} & \multicolumn{1}{l}{\textbf{Definition}} \\ \hline 
\endhead

\hline \multicolumn{3}{r}{{Continued on next page.}} \\ \hline
\endfoot

\hline
\endlastfoot

1 & Slots & A permission that is given by a coordinator for a planned operation to use an airport's infrastructure at a specific date and time. \\ 
2 & Airport Level & There are three airport levels: (1) for those whose demand does not exceed the infrastructure capacity; (2) for those whose demand causes certain congestion at specific times (e.g., hours, days, days of week), but which can be resolved through mutual agreements between the facilitator and air operators; and (3) for those whose demand exceeds the airport capacity limits, requiring a coordinator for the slots allocation, using WASG's best practices and prioritizing allocations for better market competitiveness. \\
3 & Infrastructure capacity and airport parameters & Compiled of the necessary parameters for the coordination of slots. Through these, it is possible to identify the operational capacity for allocation that does not exceed the demand limit, providing an adequate service level (e.g., maximum operations per hour / 30 min. / 15 min. / 5 min. on the runway, number of aircraft parking positions available on the apron, number of available gates, airspace limitations, environmental limitations, etc.). \\
4 & Seasons & There are two specific periods where operations take place. The (1) Summer Season, which starts on the last Sunday of March, and the Winter Season, which starts on the last Sunday in October. \\
5 & Series of slots & A minimum of 5 slots allocated for approximately the same time on the same day of the week through a season. \\
6 & Slot pool & All slots that will receive allocation priority at level 3 airports, after the historical slots are properly allocated. \\
7 & Historic Slots & Slots with operating precedence at the allocated airport, acquired by a regularity above 80\% in the equivalent previous season. \\
8 & Facilitator & The one responsible for collecting data and adjusting movement at level 2 airports. \\
9 & Coordinator & The one responsible for data collection and coordination of slots at level 3 airports. \\
10 & Activity Calendar & Deadlines and Events that manage the process of coordinating movement and slots for each season. It is established two per year. \\
11 & Previous Equivalent Season & Last Season of the same name, i.e., if the current season is a summer season, the previous equivalent season is the prior summer season. \\
12 & New entrants & Air operators that have a small number of operations (less than 7) for each day of a season, or companies that do not operate yet and are requesting operations. \\
13 & Annual movements & Are all movements that have specific times in both Summer and Winter seasons. \\
14 & Messages & Reference to data shared between coordinator, facilitator and air operators \cite{SSIM2020}. The messages are usually standardized in a format recognized by the industry (e.g., SSIM), containing critical information about the slot, i.e., operations' start date, operations' end date, air operator IATA's code, equipment type (e.g., Boeing 737 MAX), number of seats offered, type of movement air (e.g., Passenger or Cargo), frequency of operation during the week, hours of operation (slots), among others.\\ \hline
\end{xltabular}

\end{center}

\subsubsection{Airport level definition}

At Level 1, due to the availability of adequate schedules for every demand, the allocation becomes free of priorities.

At Level 2, the following allocation priority is applied: (i) movements that operated regularly in the previous equivalent season; (ii) annual movements, i.e., those which have operated with high regularity in the immediately preceding season; (iii) movements with more schedule operations for the current season; (iv) temporary operations; (v) movements restricted by operational factors; and (vi) any other type of movement. At Level 2, as well as a Level 3, a waiting list can be instituted for already congested periods.

For Level 3, WASG proposes a mechanism that grants access to the high-demand market, opening opportunities for new operators to obtain slots at coveted times. For these, the allocation is made prioritizing a series of slots, temporary slots, and other operations. The series of slots can be allocated prioritizing: (i) historical slots with no changes, with changes other than in coordination parameters, or with changes that impact the coordination parameters within a flexibility range period; (ii) slot pool, containing a consistent division of the remaining slots for new entrants and non-new-entrants operators; and (iii) other movements. It should be noted that additional allocation factors are commonly applied to the Level 3 slot pool, such as annual movements within a flexibility range, movements with more operations for the current season, movements that are restricted to certain schedules due to regulation, movements with longer time at waiting lists, movements that have a specific type of service (e.g., passengers and cargo) or market (e.g., regional, long-distance and international destinations), movements with the greater number of frequencies, among other factors established by a stakeholders' committee.

Each airport level has a different allocation priority. The \refFig{fig_pri_alloc} illustrates the allocations by airport level, specifying the prioritized movements.

\figuraBib{3_allocation_en}{Allocation priority for each airport level}{}{fig_pri_alloc}{width=0.8\textwidth}%

\subsubsection{Slot allocation: calendar activities}

The slot allocation process, defined by the WASG activity calendar, has the following steps: 

\begin{enumerate}
    \item The calendar is published by the regulator for the immediately following season;
    \item Airports must declare their capacity and coordination parameters within 7 days of initial submission;
    \item Coordinators must provide details of historic slots through standard messages (SHL);
    \item Air operators must review the historic slots received (AHD), contacting the coordinators if they disagree with the SHL, which results in an open communication between stakeholders;
    \item The air operators send the initial submission with the requested slots (ISD);
    \item The coordinators inform the air operators of the result of the initial submission, i.e., the airport's network after the distribution of demand (SAL);
    \item Coordinators and air operators enter into dialogue to adjust offers for more advantageous slots, which can be done in a conference (SC), where exchanges of slots can occur between air operators, when local regulations allow them and when it is confirmed the advantage by the coordinator;
    \item After the conferences, a continuous reallocation process begins until the limit for the return of slots (SRD), where any movement that will not be operated must be canceled;
    \item With the reinstate of non-operational slots, the reference base (BDR) is created on which the 80\% regularity will be based; and
    \item The last step takes place through operational planning, where the airline community plans the airspace, stands, and gates, among others infrastructure need to be prepared to receive the movements.
\end{enumerate}

After all the steps, new slots can be requested if feasible allocation. The season starts with the first planning operation on the last Sunday of October (Winter Season) or on the last Sunday of March (Winter Season). Simultaneously, the planning for the immediately following season takes place. The \refFig{fig_temp} illustrates the season system, with each season indicated by specific colors. Note that during the activities of a specific season, the activities related to the immediately following season are already started, in a continuous process.

\figuraBib{2_temporadas_en}{Illustration of the calendar of activities stipulated by the WSGA}{}{fig_temp}{width=1\textwidth}%

\subsubsection{Slot Monitoring}

For the slots monitoring, WASG recommends that it should be divided into (i) pre-operation analysis, which identifies the slots misuses before the day when the movements occur; and (ii) post-operation analysis, which identifies whether a misuse has occurred and whether the operator has achieved the historic right in its slots. The objective is to ensure proper operation, avoid the waste of airport infrastructure, and open communication channels between stakeholders.