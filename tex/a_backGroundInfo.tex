
\label{section:Introduction}

\section{The continuous growth of the air transportation industry}%

The air transportation industry has been growing since 1945. It passed through several economic and geopolitical crises, the most recent being the COVID-19 pandemic. The latter has significantly impacted the world passenger traffic, reducing the industry by 40\%, when comparing 2021 against 2019, before the spread of the virus. This diminish of the sector cost approximately 324 billion dollars of gross passenger operating revenue for airlines (\citealp{ICAOEconomicImpact}). The COVID-19 also influenced the airport's revenue and operations. At the beginning of the pandemic, the close sky policies allowed only essential flights, reducing the use of the airport's infrastructure. Sanitary measures also changed the aircraft movement itinerary, due to mandatory cleaning and inspections before and after transporting passengers, which directly affected the demand that could be accepted by the airports. Even through all these negative indicators, it is estimated a recovery of almost 103\% of worldwide passenger transportation in 2024, when comparing against 2019 (\citealp{IATA2024}). Some regions, such as Central America, have projections that surpass the passenger demand of 2019 at the end of 2023. Therefore, even with this uncertainty, the increasing demand required by the airlines, to compose a better network, will result in congestion of several airports, thereat old discussions of how to efficiently resolve the distribution of aircraft operations at airports come back into evidence.

\subsection{Airport Conception}

In order to better understand this study's main problem, it is important to have knowledge of the basic concepts of airport’s operation. The prototypical airport is composed of 2 parts: an (i) airside and (ii) landside (\citealp{Lance2009}). From an operational airport’s perspective, the aircraft movement starts at the airside, when it approaches the runway through the commands of the Air Traffic Management (ATM). After the landing at the runway, the aircraft continues the movement, now at the ground level, through the taxiways. The aircraft approaches the apron in the direction of the parking stand. When it is positioned at its designated stand, the aircraft parks and bridges or airstairs are attached to the equipment in order to disembark the passengers, leading them to a gate. After the gate, the passengers are at the landside, in the airport terminal. There, they may enjoy the commercial stores (e.g., convenience stores, restaurants, and car rental agencies) before collecting their luggage and leaving the airport by ground transport or connecting to other flights. 

The opposite movement is also possible: the passengers arrive at the airport using ground transportation, they may also use the airport’s parking lots; they access the terminal, after check-in procedures and security inspections; they may enjoy the commercial stores, before embarking to the aircraft; at the airside, the aircraft pushes back, after it gets disattached from the airport bridges or airstairs; it proceeds through the taxiways until it arrives at the runways; and finally it takes-off, ending its movement at the airport. 

With the prototypical airport conception, it is possible to start the discussion about two main concepts: (i) how the airport generates revenue, and (ii) what are the main limitations of the airport infrastructure. These two are main components of the airport’s slots allocation problem.

\subsection{Airport Revenue Generation}

The airport revenue is mainly generated through aeronautical and non-aeronautical sources, both of which are directly related to the aircraft traffic volume. The former is composed by: the fare paid by each passenger using the airport infrastructure, usually included in the flight cost paid to the airlines; and the movement charge, which is paid by the airline when it lands, uses the infrastructure (e.g., hangar and terminal), and takes-off from the airport. The latter is the revenue generated by: the commercial concession at the terminal, such as rental fees for the retail at the airport area; the real estate exploitation, which includes hotels, restaurants, and parking lots at the airport perimeter; and services offered to passengers and airlines. A market equilibrium between the two revenues sources is needed, since reduction of the aeronautical charges commonly means stimulation to the non-aeronautical revenues, through passenger demand increments, but it does not guarantee the a overall maximum return (\citealp{ICAO2013}).

In Europe (2019), it is shown by \citeauthor{ACI2021} (\citeyear{ACI2021}) that 54\% of the overall airport revenue was generated by aeronautical sources. \citeauthor{ICAO2013} (\citeyear{ICAO2013}) points out that 63\% of the  aeronautical income comes from charges directly applied to passengers. The article also argue that airport users do not pay the full cost of the infrastructure the use, explaining that, in 2012, 69\% of the airports worldwide were loss-making. Seeking to profitability, the airports have increased the variety of commercial activities. It can cushioning the impact of lower passenger and freight volumes, due to its higher profit margins. This creates a cross-subsidization between aeronautical and non-aeronautical revenues sources, since profits from commercial exploitation, for example, can be reinvested in airport infrastructure, increase the demand of passenger or aircraft that can be accepted at the airport, reducing capital needs and overall cost.

Knowing that airport are important for a country social-economic-growth and, as it was said, with an efficient management, it can achieve profitability, governments have decided that, under the right economic conditions, they can included private sector participation (e.g., outright ownership, short or long-term concessions, public-private-partnership schemes, and management contracts) for the financing and operation of the airport infrastructure. The airports, under this new ownership model, have sought for better ways to compete for both passenger and airlines. The aviation community agrees with the need of infrastructure investments to accommodate the growth of the industry, increasing the profit margins. Since most of the infrastructure added are in large increments, and long time to plan are needed for these ventures, the airports are exposed to considerable risk (\citealp{ICAO2013}). Even with the constant pressure for the airport expansion, most of the administrator have to better manage its current infrastructure, until the investment risks are reduced to an acceptable level.

\subsection{Airport's infrastructure capacity}

The limitation of the airport’s infrastructure use is related with (\citealp{ARC2021}; \citealp{WASG2020}):
\begin{enumerate}
	\item How many aircraft movements the airport’s runways can accept in a certain period (e.g., one hour, 30 minutes, 15 minutes, and 5 minutes);
	\item How many stands are available at the airport, and what are their category (e.g., cargo or passenger commercial flights);
	\item How many gates are available for the passenger movements; 
	\item How many passengers can be processed through the security inspection at the check-in;
	\item How many passengers are acceptable for the terminal area available; and
        \item How many aircraft can use the infrastructure through time slot and environmental restrictions.
\end{enumerate}%

The most restrictive infrastructure capacity can be used to accept movements at the airport, or a combination of several limitations can be best suited for the task to distribute slots without damaging the airport’s level of service, such as movement delays and congestion at the taxiways. 


\section{Slot allocation process}%

\subsection{Problem statement}
\subsection{Systematic Literature review}