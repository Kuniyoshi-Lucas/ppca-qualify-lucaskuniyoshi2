\section{Planning the review}
\label{section:Planning}

To gain a comprehensive understanding of the main problem and existing approaches, a \acrfull{SLR} was conducted following a transparent process to minimize bias and enhance reliability.

The \acrshort{SLR} adhered to a three-step methodology: (i) Planning, this stage involved defining the research need, translating research questions into specific search terms, and developing a detailed review protocol prior to data collection; (ii) Conducting the Review, this stage encompassed identifying relevant research addressing the slot allocation problem, selecting primary studies, assessing their quality, extracting key data, and synthesizing the findings; and (iii) Presenting the Review Results, this stage aimed to summarize the key research findings related to the central problem and demonstrate how the proposed methods in this paper can address identified gaps in the existing literature.

A thorough understanding of airport slot allocation necessitates a comprehensive literature review to identify, analyze, and interpret diverse perspectives on the problem and its existing solutions. This review provides historical context and clarifies the potential contributions of this paper. To ensure a bias-mitigated and transparent data collection process, the review criteria and procedures were meticulously planned before execution, adhering to the guidelines outlined in \citeonline{wohlin2012experimentation} and \citeonline{shamseer2015prisma}. \refFig{fig:experiment} illustrates the sequential processes and their respective deliverables.

\figuraBib{10_literaturaReview_en}{The literature review processes and their deliverables. Adapted from: \cite{wohlin2012experimentation}}{}{fig:experiment}{width=0.6\textwidth}%

\subsection{The need for a review}

The first step was the identification of the need for a review, i.e., understand how the problem has been addressed in past research and what are the previous limitation. 

Airport infrastructure exhibits significant variability, encompassing distinct configurations of runways, taxiways, gates, and terminals. Furthermore, airports operate within diverse environments, including densely populated areas, which significantly influences demand. These unique characteristics can lead to diverse perspectives on the slot allocation problem and consequently, a range of proposed solutions. Given the well-established nature of the slot allocation problem, a thorough understanding of its historical management is crucial.


While a \textit{snowballing} approach could potentially gather relevant literature, it carries the risk of introducing search bias, potentially limiting the comprehensiveness and reproducibility of the review \cite{jalali2012systematic}. A systematic review offers a more rigorous approach, ensuring a comprehensive understanding of the problem through a well-defined, step-by-step protocol that is subject to review and potential improvement.

This review assumed the absence of prior systematic reviews on this specific topic. However, if any such reviews were encountered, their contributions and limitations would be carefully evaluated

\subsection{The \acrshort{SLR} researches questions}

The second step involved formulating specific research questions to guide the literature search. These research questions serve as crucial focal points for identifying relevant primary studies and extracting and analyzing relevant information. To ensure clarity and focus, the PICO framework was employed to systematically break down each research question into its essential components: Population, Intervention, Comparison, and Outcome.

Furthermore, the 'Intervention' component was specifically defined to clarify the approach of the present paper and determine whether similar approaches have been previously utilized in the literature.

The following research questions were formulated for this literature review:

\begin{itemize}
    \item \textbf{RQ1:} In medium and large airports (P), how do integer programming, queue theory, and machine learning techniques (I)  compared to the \acrshort{IATA}'s guidelines for the administrative management of slots (C) in terms of their effectiveness in reducing the number of changes to airlines' initial allocations and optimizing airport infrastructure utilization (O)?
\end{itemize}

To further refine the inquiry, the following three sub-questions were formulated: (i) What methodologies have been employed to solve the airport slot allocation problem? (ii) What are the primary constraints associated with these approaches? (iii) Are there any identified gaps or underlying assumptions that warrant further exploration? It aims to collect the most valuable information about the models created to solve the problem and how experimentation with algorithms are helping them to tackle it.

\subsection{Review protocol}

The third and final step in the planning phase involved the development of a comprehensive review protocol. This protocol serves as the foundation for the current research review and plays a crucial role in ensuring the reproducibility and validity of the study. The following key elements were considered during the development of the review protocol:

\subsubsection{Search strategy for primary studies}

The primary activities within this step involved:

\begin{itemize}
    \item \textbf{Formulating a comprehensive search string:} this was achieved by translating the research questions into a set of relevant keywords.
    \item \textbf{Executing database searches:} the formulated search string was then applied to a range of relevant academic databases.
\end{itemize}

The search string was constructed using a combination of keywords, including: "Airport", "Slot", "Allocation", "Integer Programming", "Queue Theory", "Machine Learning", "Optimization", and "IATA Slot Guidelines".

To refine the search, boolean operators were employed to effectively combine these keywords, resulting in the following search string:
\\

\textit{(airport* \textbf{AND} "slot allocation" \textbf{AND}
 optimization \textbf{AND} ("IATA guidelines" \textbf{OR} "Worldwide Airport Slot Guidelines") \textbf{AND} ("machine learning" \textbf{OR} "integer programming" \textbf{OR} "queue theory")) \textbf{AND} [E-Publication Date: (01/01/2002 \textbf{TO} 01/12/2023)]}. 
\\

\subsubsection{Search sources}

The formulated search string was subsequently employed to systematically search the following academic databases to identify relevant primary research studies:
\begin{enumerate}
    \item ACM Digital Library;
    \item \acrfull{BASE};
    \item EBSCO;
    \item ScienceDirect;
    \item Scopus;
    \item \acrfull{TRID};
    \item Web of Science;
    \item \acrfull{DBLP};
    \item \acrfull{DOAJ};
    \item IEEE Xplore;
    \item JSTOR; and
    \item Springer Link (Article section).
\end{enumerate}


The selection of databases was guided by two primary criteria: (i) Multidisciplinary Coverage, the databases were chosen to encompass both multidisciplinary subjects and those specializing in computer science or transportation studies, aligning with the core research areas of this paper; and (ii)  Search Capability, the databases were selected based on their comprehensive coverage and robust search capabilities, as outlined in \citeonline{kitchenham2007guidelines}, this selection criterion prioritizes databases that facilitate systematic searches through the utilization of queries and filters, ultimately ensuring the reproducibility of the search process with high recall and precision.  

Following the criteria established by \citeonline{gusenbauer2020academic}, databases were classified as 'primary' if they met all necessary quality requirements, including the possibility of query refinement, string size criteria, server response criteria, use of boolean operators, functionalities and reproducibility. Databases 1 to 7 were designated as primary databases, while the remaining databases were considered supplementary. No additional search strategies (e.g., searching specific journals and conference proceedings or contacting researchers to obtain unpublished material or grey literature) were employed.

This step necessitates the meticulous removal of all identifiable false positives and duplicate entries across the searched databases.

To validate the search string's effectiveness, the article by \citeonline{ribeiro2018optimization} was used as a benchmark. This article, previously identified as relevant to the research scope, must be retrieved within the search results from at least one of the targeted databases. Failure to retrieve this article necessitates a reformulation of the search string.

\subsubsection{Study selection criteria}
To mitigate bias, specific inclusion and exclusion criteria were established prior to the screening process. Employing the "First Pass" approach as established by \citeonline{keshav2007read}, all primary papers were screened by carefully reviewing their titles, abstracts, introductions, and conclusions

\noindent \textit{Inclusion criteria:}

\begin{enumerate}

    \item The study must address any aspect of the airport slot allocation problem, encompassing models, perspectives, constraints, methods, technical considerations, etc.; and
    \item The study must propose a solution for the airport slot allocation problem utilizing integer programming, queue theory, or machine learning techniques.
\end{enumerate}

\subsubsection{Study exclusion criteria}

\textit{Exclusion criteria:}
\begin{enumerate}
    \item \textbf{Duplicates and Old Versions}, articles are excluded if they are duplicates of other included articles or represent older versions of the same study. If an article appears in multiple primary and/or secondary databases, it is retained in the database listed first in the "Search Sources" section and removed from the remaining databases.
    \item \textbf{Publication Date}, articles published before 2001 are excluded. This timeframe is chosen to focus on research conducted after the significant market regulations implemented following the September 11th attacks.
    \item \textbf{Scope of Slot Guidelines}, articles introducing slot allocation guidelines beyond those outlined in the \acrshort{IATA}'s Worldwide Slot Guidelines (\acrshort{WASG}) \cite{WASG2020} are excluded.
    \item \textbf{Focus on Auction-based Solutions}, articles that primarily focus on auction-based economic approaches to solving the airport slot allocation problem are excluded.
    \item \textbf{Lack of Full-Text Access}, articles that are not accessible in full-text format are excluded.
    \item \textbf{Irrelevant Scope}, articles that are not directly relevant to the research questions and do not address the core slot allocation problem, but instead focus on adjacent areas such as ground delays or weather impacts on airports, are excluded.
    \item \textbf{Specific Context}, articles that focus on specific contexts, such as the September 11th attacks or the COVID-19 pandemic, are excluded to maintain a broader and more generalizable perspective.
    \item \textbf{Language}, articles not published in English or Portuguese are excluded.
\end{enumerate}

\subsubsection{Study selection procedures}

The removal of all irrelevant papers was conducted through a thorough assessment of each study against the established selection and exclusion criteria. This process involved a careful review of titles, abstracts, and conclusions, with a focus on identifying and excluding papers that did not meet the inclusion criteria. If more than one section of the paper (title, abstract, or conclusion) was deemed irrelevant to the scope of this review, the paper was excluded, and a concise justification for its exclusion was recorded.

\subsubsection{Narrative Synthesis}

 Given the anticipated variability and potential differences among the included studies – stemming from the application of diverse methodologies across varied samples (e.g., different airports) and potentially leading to methodological heterogeneity - this review will employ a narrative synthesis approach to synthesize the extracted data and assess the overall body of evidence \cite{campbell2018improving, campbell2019lack}. 
 
 This method is particularly suitable when the heterogeneity of the included studies precludes statistical meta-analysis \cite{popay2006guidance, campbell2018improving, campbell2019lack}. 
 
 Drawing upon the guidance for Narrative Synthesis outlined by \citeonline{popay2006guidance}, this study will undertake the following key stages: 
 \begin{enumerate}
     \item \textbf{Developing a theoretical understanding of the problem:} This stage will involve constructing an initial framework or theory of how the identified software problem is conceptualized and addressed within the existing literature, considering the different approaches and contexts. 
     \item \textbf{Developing a preliminary synthesis of findings through tabulation:} A systematic tabulation of key characteristics and findings from each included study will be created. This will facilitate an initial overview and comparison of the evidence.
     \item \textbf{Exploring relationships within the data using concept mapping:} Concept mapping techniques will be employed to visually represent and explore the relationships between different concepts, interventions, outcomes, and contextual factors identified across the studies. This will aid in identifying patterns and potential associations. 
     \item \textbf{Assessing the robustness of the synthesis through critical reflection:} The final stage will involve a critical discussion of the synthesis process, considering the strengths and limitations of the included studies (as determined by the quality assessment), the potential for bias, and the overall confidence in the synthesized findings. This reflection will explicitly address the impact of methodological heterogeneity on the conclusions drawn.
 \end{enumerate}

\section{Conducting the review}
\label{section:Conducting}
This section presents the outcomes of the systematic search and selection process, including a comprehensive review of the validity checks implemented throughout the study. These checks encompassed data extraction procedures and a rigorous quality assessment of the included studies.

\subsection{Identification of research}

The initial search yielded 31 sources across 12 databases. After removing 9 duplicates (29.03\%), a total of 22 unique sources remained for further analysis, as detailed in Table \ref{table*:results}.

\begin{table*}[h]
\begin{center}
\caption{Search results per database}
\small
\begin{tabular}{*{15}{l}}
\hline
\multicolumn{1}{l}{\#} & \textbf{DATABASE}                                     & \multicolumn{1}{l}{\textbf{SOURCES FOUND}} \\ \hline
1  & ACM Digital Library                      & 20 \\
2  & Bielefeld Academic Search Engine (BASE)  & 8  \\
3  & EBSCO                                    & 20 \\
4  & ScienceDirect                            & 74 \\
5  & Scopus                                   & 8  \\
6                      & Transport Research International Documentation (TRID) & 8                                          \\
7  & Web of Science                           & 6  \\
8  & CiteSeerX                                & 0  \\
9                      & Digital Bibliography \& Library Project (DBLP)        & 8                                          \\
10 & Directory of Open Access Journals (DOAJ) & 3  \\
11 & IEEE Xplore                              & 1  \\
12 & JSTOR                                    & 1  \\
13 & Springer Link                                         & 19                     \\ \hline
   & \textbf{TOTAL}                                        & 146                    \\ \hline
\end{tabular}
\label{table*:results}
\end{center}
\end{table*} 

All studies were identified within the designated primary databases, with ScienceDirect emerging as the most prolific source, contributing 59.09\% of the total retrieved articles. Notably, three databases – \acrshort{DBLP}, \acrshort{DOAJ}, and IEEE Xplore – did not yield any results in the initial search, even when considering duplicates.

To ensure a comprehensive understanding of the relevant literature, a rigorous screening process was implemented. Following the "first pass" approach outlined by \citeonline{wohlin2012experimentation}, each study's title and abstract were independently and randomly reviewed. Studies were excluded if they did not directly address the airport slot allocation problem or failed to meet the established inclusion criteria. This screening process resulted in the selection of 19 primary studies, representing 86.36\% of the unique sources identified.
\\

For the exclusion criteria: 
\begin{enumerate}
    \item One article was not accessible;
    \item One article was reviewing the slot allocation problem through economic approaches; and
    \item One article was focusing in adjacent area (flight planning optimization).
\end{enumerate}

Following the selection procedures, the systematic literature review (SLR) yielded 19 primary studies. Notably, 13.63\% of the initial search results were identified as false positives, i.e., studies retrieved by the search string but ultimately deemed irrelevant to the research scope. Table \ref{table*:falsepositive} provides a summary of the primary studies identified per database, including the number of false positives observed after duplicate removal.


\begin{center}
    
\setlength{\tabcolsep}{10pt} % Default value: 6pt
\renewcommand{\arraystretch}{1.5} % Default value: 1
\begin{xltabular}{\textwidth}{p{0.5cm} p{7cm} p{3cm} p{2.6cm}}
\caption{Primary studies per database and the false positive results.} \label{table*:falsepositive} \\

\hline \multicolumn{1}{c}{\#} & \textbf{DATABASE}                                     & \textbf{PRIMARY STUDIES} & \textbf{FALSE POSITIVES} \\ \hline 
\endfirsthead

\multicolumn{4}{c}%
{\tablename\ \thetable{} - Continued from previous page.} \\
\hline \multicolumn{1}{c}{\#} & \textbf{DATABASE}                                     & \textbf{PRIMARY STUDIES} & \textbf{FALSE POSITIVES} \\ \hline 
\endhead

\hline \multicolumn{4}{r}{{Continued on next page.}} \\ \hline
\endfoot

\hline
\endlastfoot

1                    & ACM Digital Library                            & 4  & 80.00\%     \\
2                    & Bielefeld Academic Search Engine (BASE)        & 4  & 50.00\%  \\
3                    & EBSCO                                          & 3  & 83.33\%  \\
4                    & ScienceDirect                                  & 30 & 58.33\%  \\
5                    & Scopus                                         & 0  & -   \\
6                      & Transport Research International Documentation (TRID) & 2                                            & 0.00\%                   \\
7                    & Web of Science                                 & 0  & -        \\
8                    & CiteSeerX                                      & 0  & -        \\
9                    & Digital Bibliography \& Library Project (DBLP) & 1  & 66.67\%  \\
10                   & Directory of Open Access Journals (DOAJ)       & 1  & 33.33\%  \\
11                   & IEEE Xplore                                    & 0  & -        \\
12                   & JSTOR                                          & 0  & - \\
13                   & Springer Link                                  & 2  & 88.89\%  \\ \hline
\multicolumn{1}{l}{} & \textbf{TOTAL}                                          & 47 & 67.81\%  \\ \hline
\end{xltabular}

\end{center}

The distribution of studies by year of publication is visualized in \refFig{fig:yearPublication}. Notably, the subject of this investigation has gained significant attention in recent years, with 43.37\% of the selected sources published between 2020 and 2023. The selection process is illustrated in \refFig{fig:resumeSLR}.

\figuraBib{12a_yearPublication}{Number of studies by year of their publication}{}{fig:yearPublication}{width=0.8\textwidth}%

\figuraBib{11a_SLR_brief}{The studies selection procedures and deliverable. Adapted from: \cite{page2021prisma}}{}{fig:resumeSLR}{width=1.0\textwidth}%


\section{Review report process}

\subsection{Theoretical understanding}
Initially, it is feasible to approach the review by synthesizing multiple pieces of evidence extracted from the individual studies included, in order to construct a model that highlights key concepts or issues pertinent to the review question and delineates the relationships among them \cite{popay2006guidance}.

The air transport industry, encompassing airlines, airports, and local regulatory agencies, has sought methods to utilize slot allocation coordination as a means to optimize the available airport infrastructure. This approach presents a viable solution to address the rapid increase in air transport demand in the short term. Consequently, two primary questions arise: (i) how can slot distribution be optimized to align with the airport's available infrastructure, and (ii) what are the constituent elements of the airport's operational capacity?

Stakeholders are pursuing a model that facilitates the efficient utilization of airport infrastructure, reduces delays, enhances the probability of airlines retaining allocated slots, and ensures the necessary equity in the distribution of these scarce resources. The long-term impact of such a model includes a more comprehensive understanding of airport limitations, as well as the improvement of the financial performance of all stakeholders and the current slot regulations. 

The figure \refFig{fig:narrative_one}.provided illustrates the theoretical framework of the problem, progressing from the desired impact and its associated benefits and metrics, to the principal activities required to initiate the model for a more effective resolution of the initial problem (theory of change).

The validity of the initial understanding will be demonstrated within the following sections of the report. The aim is to answer the research questions posed by the systematic literature review

\figuraBib{13_narrative_section1}{Theoretical understanding of the problem: Linking slot optimization and airport sensitivity to enhanced allocation and financial outcomes. Adapted from: \cite{popay2006guidance, eduspots_theoryofchange}}{}{fig:narrative_one}{width=1.0\textwidth}%


\subsection{Preliminary synthesis of findings through tabulation}

To synthesize the information derived from the 19 selected studies, the Table \ref{table*:SRL_narrative_table} was constructed. The objective of this table was to systematically present each study, detailing its title, applied methodologies, population under investigation, and principal findings. The numerical designation assigned to each article will serve as a consistent reference throughout this report.

Within the table, in addition to the primary methodologies employed in each study, the applicability of each model was determined based on the coverage classification—namely, whether it pertained to a single airport or an airport network \cite{zografos2017increasing}. Furthermore, to facilitate a more comprehensive understanding of the approaches and to underscore the heterogeneity in models, data, and results across the cases, the specific population of each study is also indicated (e.g., the name of the airport or the number of airports comprising the airline network). The main results of each study were explicitly stated in the table, albeit not exhaustively, encompassing both implicit and explicit outcomes and contributions of the respective models.

\clearpage

\begin{sidewaystable}

\caption{Primary studies collection} \label{table*:SRL_narrative_table}
\tiny
%\begin{tabular}{llllll}
\begin{tabular}[H]{p{0.25cm} p{3cm} p{5cm} p{3.5cm} p{3.5cm} p{8cm}}
\toprule
\centering

\textbf{\#} & \textbf{CITATION} & \textbf{TITLE} & \textbf{METHOD} & \textbf{POPULATION} & \textbf{RESULT} \\

\midrule

1 & \citeonline{zeng_data-driven_2021} - Research paper & A data-driven flight schedule optimization model considering the uncertainty of operational displacement &  Linear programming method (Single airport) & \acrlong{HGH} (two season) & This research demonstrates a reduction of over 60\% in both arrival and departure traffic compared to other optimization model. \\
\midrule
2 & \citeonline{ribeiro_large-scale_2019} - Research paper & A large-scale neighborhood search approach to airport slot allocation & Heuristic solution method and integer programming method (Single airport) & \acrlong{LIS} (two seasons) & The authors' algorithm produced solutions within 0.1\% of the optimum in just a few hours. \\
\midrule
3 & \citeonline{zografos2017increasing} - Integrated review & Increasing airport capacity utilisation through optimum slot scheduling: review of current developments and identification of future needs & Literature review & N/A &  This review encompassed 96 research papers published between 1993 and 2014, developing a classification system for the approaches used in the slot allocation problem.\\
\midrule
4 & \citeonline{ribeiro2018optimization} - Research paper & An optimization approach for airport slot allocation under \acrshort{IATA} guidelines & Integer programming method (Single airport)   & Madeira and Porto airports (one season)  & The model improves slot allocation outcomes by reducing the displacement experienced by airlines by an estimated 4.5\% at Madeira and 27\% at Porto. \\
\midrule
5 & \citeonline{benlic_heuristic_2018} - Research paper & Heuristic search for allocation of slots at network level & Heuristic solution method (Network of airports) & Set of benchmark (half year) & The article demonstrates that even under strict rules, a network-wide allocation method can achieve a coherent schedule that minimizes delays. \\
\midrule
6 & \citeonline{wang_distribution_2022} - Research paper & Distribution prediction of strategic flight delays via machine learning methods & Machine learning-based model (Single airport) & \acrlong{CAN} (3 years) & Prediction accuracy exceeding 0.80 was achieved. \\
\midrule
7 & \citeonline{jacquillat_interairline_2018} - Research paper & Interairline equity in airport scheduling interventions &  Lexicographic modeling (Single airport) & \acrlong{JFK} (one calendar day) & The article shows that achieving maximum equity requires no (or minimal) sacrifice in terms of efficiency losses. \\
\midrule
8 & \citeonline{zografos_bi-objective_2019} & A bi-objective efficiency-fairness model for scheduling slots at congested airports & Bi-objective modelling framework (Single Airport) &  Data resembling real world conditions of a Coordinated Airport (one season) & The paper presents a bi-objective model that combines efficiency and fairness. The model uses the $\epsilon$-constraint method, where one objective (fairness) is turned into a constraint and then iteratively tightened to explore trade-offs between fairness and efficiency.  \\
\midrule
9 & \citeonline{jiang_decision_2021} & A decision making framework for incorporating fairness in allocating slots at capacity-constrained airports &  Bi-objective modelling framework (Single Airport) & One airport (one season) & The paper presents a framework that combines efficiency and fairness. A key innovation is the incorporation of a voting mechanism. Airlines express their willingness to trade-off efficiency for fairness depending on their individual weight. \\
\midrule
10 & \citeonline{jacquillat_roadmap_2018} - Integrated review  & A roadmap toward airport demand and capacity management & Literature review & N/A & The review provides a framework that underscores the critical interdependencies between operational/managerial, and economic considerations in airport demand management. \\
\midrule
11 & \citeonline{gillen2016airport} - Integrated review & Airport demand management: The operations research and economics perspectives and potential synergies] & Literature review & N/A & Introduced an integrated framework and highlighted potential research synergies at the intersection of operations, economics, and management. \\
\midrule
12 & \citeonline{dixit_algorithmic_2023} - Research paper & Algorithmic mechanism design for egalitarian and congestion-aware airport slot allocation & A game-theoretic model and a mechanism design solution (Single airport) & \acrlong{DEL} and \acrlong{MAA} (5 calendar days) & The model yielded a 5-20\% and 20-30\% increase in social utility at \acrshort{DEL} and \acrshort{MAA}, respectively. \\
\midrule
13 & \citeonline{kuran_heuristic_nodate} - Thesis & Heuristic optimization methods for seasonal airport slot allocation &  Heuristic solution method (Single airport) & Vienna Airport (5 seasons) & The proposed method demonstrates promising results in minimizing time deviations from the requested slots. \\
\midrule
14 & \citeonline{lambelho_assessing_2020}  - Research paper & Assessing strategic flight schedules at an airport using  machine learning-based flight delay and cancellation predictions & Machine learning-based model (Single airport) & \acrlong{LHR} (10 seasons) & The proposed prediction models demonstrated high accuracy (0.79) across all evaluated approaches. \acrfull{LGBM} consistently exhibited superior performance. \\
\midrule
15 & \citeonline{keskin_optimal_2023} - Research paper & Optimal network-wide adjustments of initial airport slot allocations with connectivity and fairness objectives & Two bi-objective models (Network of airports)  & 16 coordinated and facilitated airports in Brazil (2019 summer season) & The case study results indicate that using connectivity metrics significantly affects the distribution of total displacement among airports. \\
\midrule
16 & \citeonline{liu_research_2022} - Research paper & Research on slot allocation for airport network in the presence of uncertainty & Mixed integer programming, (Network of airports) & 15 coordinated airports in China (two calendar days) & The model minimizes the sum of displacement costs and delay costs under the worst-case scenario. \\
\midrule
17 & \citeonline{wang_slot_2023} - Research paper & Slot allocation for a multiple-airport system considering airspace capacity and flying time uncertainty & Chance constrained programming model (Network of airports)& Five airports within the Guangdong-Hong Kong-Macao Greater Bay Area (one calendar day) &  It outperforms those from the certainty model and the original schedule with the cost of a small number of increased slot displacements. \\
\midrule
18 & \citeonline{pellegrini_sosta_2017} - Research paper & SOSTA: An effective model for the simultaneous optimisation of airport slot allocation & Integer linear programming model (Network of airports) & European Level 2 and Level 3 airports (day with the highest air traffic volume of 2013) & It proved valuable in conducting a comprehensive sensitivity analysis of various model parameters and objective functions. \\
\midrule
19 & \citeonline{corolli_time_2014} - Research paper & The time slot allocation problem under uncertain capacity &  Two stochastic programming models (Network of airports) & Multiple European airport network datasets (4 calendar days) & The models can reduce schedule/request discrepancies and operational delays by up to 58\%. \\
\bottomrule
\end{tabular}
\end{sidewaystable}
\clearpage

\subsection{Exploring relationships within the data: Concept Mapping}

To address the information presented in each of the studies, by exploring their interrelationships through the topics of motivation, applied methods, obtained results, and limitations, thus illustrating the heterogeneity among them, the present systematic literature review developed a conceptual map.

The simplified map, illustrated in the \refFig{fig:narrative_two}, presents four subdivisions for enhanced comprehension: (i) Motivation for Improvement, wherein the principal reasons for the development of studies aimed at improving slot allocation at airports will be expounded and interrelated; (ii) Methods, wherein the approaches will be explained, along with their continuous developments and enhancements; (iii) Results, wherein, given the applied methods, the main contributions obtained are presented; and (iv) Limitations, wherein aspects not fully developed within the respective research, those indicated for future work, and gaps in the methodological approaches are elucidated.

The relationships among the studies within each subdivision will be elaborated upon in subsequent subsections. It is pertinent to note that the numerical designation of each study, as presented in the Table \ref{table*:SRL_narrative_table}, will be employed to simplify the connections between them.

\figuraBib{14a_narrative_section3_total}{Simplified conceptual mapping of the findings from Slot Allocation Problem \acrlong{SLR}}{}{fig:narrative_two}{width=0.4\textwidth}%

\subsubsection{Concept Mapping: Motivations for improvement}

Three principal branches originate from the motivations identified among the selected studies: (i) the anticipated increase in air traffic, attributed to the growing number of global passengers; (ii) the balance between efficiency and fairness in slot allocations, representing a pursuit of optimal infrastructure distribution that maximizes its utilization, alongside an approach ensuring that the scarce resource is allocated with transparency, non-discrimination, and equality; and (iii) the inherent complexity of the slot allocation problem, given its combinatorial nature. The \refFig{fig:narrative_three} illustrates the three main branches and their sub-branches, depicting motivations that are a result of the former.

The increasing development of the civil air transport industry is consistently identified across all studies as a primary reason to enhance the efficiency of slot management methods. This factor, combined with the current capacity constraints of airports – whether due to infrastructure limitations or regulations – leads to congestion. This congestion is responsible for widespread delays within airline networks, which can generate costs for all stakeholders in the industry \cite{ribeiro2018optimization, dixit_algorithmic_2023}.

Prior to the COVID-19 pandemic, global passenger volume was projected to reach 11.4 billion by 2024, representing 124\% of the 2019 level. Current projections for 2024 estimate a global passenger volume of 9.5 billion, which is 104\% of the 2019 level, indicating a 9\% year-over-year growth from the 2023 volume \cite{ACI_AirTravelDemand2025}. Consequently, despite the impact of the pandemic on the aviation industry, persistent congestion is anticipated at airports worldwide due to this continued growth in demand. This congestion often results in decreased flight punctuality and a diminished passenger experience \cite{liu_research_2022}.

\figuraBib{14b_narrative_section3_motivation}{Main motivations identified in the studies for addressing the airport slot allocation problem. The articles are referenced by the numbers presented in Table \ref{table*:SRL_narrative_table}}{}{fig:narrative_three}{width=0.7\textwidth}%

The selected studies highlight the intensifying imbalance between rapidly growing air traffic and limited airspace, leading to reduced flight punctuality and increased delays \cite{corolli_time_2014, gillen2016airport, pellegrini_sosta_2017, jacquillat_interairline_2018, ribeiro2018optimization, jacquillat_roadmap_2018, zografos_bi-objective_2019, lambelho_assessing_2020, zeng_data-driven_2021, kuran_heuristic_nodate, keskin_optimal_2023, wang_slot_2023}. This situation underscores the critical need for effective slot allocation strategies to mitigate these adverse effects.

In 2019, the number of flights operated by passenger airlines in mainland China reached 4.611 million, a 6.1\% increase compared to the previous year \cite{wang_distribution_2022}. These authors, citing \citeonline{CAAC_StatisticalBulletin2019}, indicate a trend of increasing air passenger traffic in China. The average delay per passenger flight in 2019 was 14 minutes, representing a slight decrease compared to 2018. Such operational restrictions can lead to lost or displaced demand, negatively impacting the total revenue for stakeholders \cite{jacquillat_roadmap_2018}

\citeonline{benlic_heuristic_2018} shows a projection that indicates that air traffic in Europe will increase to 14.4 million flights by 2035. Consequently, in their most likely scenario, approximately 12\% of demand will be unaccommodated, leading to delays.

\citeonline{corolli_time_2014} and \citeonline{zografos2017increasing} highlight that the rapid growth of air transport services in Europe, coupled with institutional limitations on the construction of new airports, has resulted in congestion issues manifested by increasing delays. In 2016, the average delay per flight within the European Civil Aviation Conference area was 18 minutes \cite{Eurocontrol_AnnualReport2016}. Furthermore, the authors note that \acrfull{ATM} inefficiencies in the European Union caused 10.8 million minutes of flight delays in 2012, incurring costs of €4.5 billion for airspace users and €6.7 billion for passengers, and generating 7.8 million tonnes of excess CO2 emissions \cite{IATA_SESFactsheet2014a}.

\citeonline{ribeiro2018optimization} cited data on flight punctuality in the United States and Europe. According to \citeonline{EUROCONTROL_USEuropeATM2024}, 80.1\% of flights arriving in the U.S. were within 15 minutes of their scheduled time, whereas in Europe, this figure was 76.5\%.

\citeonline{ribeiro_large-scale_2019} assert that the growth in air traffic demand has surpassed the available capacity at numerous airports globally, leading to the frequent occurrence of flight delays and substantial costs for airports, airlines, and passengers. To illustrate this assertion, the authors cite the financial impact of air traffic congestion in the United States, which exceeded \$30 billion in 2007 \cite{gillen2016airport, jacquillat_interairline_2018, jacquillat_roadmap_2018, ribeiro_large-scale_2019} and reached \$33 billion in 2019 \cite{dixit_algorithmic_2023}.

\citeonline{jacquillat_interairline_2018} and \citeonline{zografos_bi-objective_2019} demonstrated that a primary objective in addressing the slot allocation problem is to balance the preferences and requirements of diverse stakeholders. This often involves navigating trade-offs between efficiency (maximizing the aggregate utility of stakeholders), equity (fairly distributing utilities among stakeholders), and potentially other objectives such as maximizing outcome predictability and ensuring incentive compatibility. This perspective has led authors to propose models incorporating inter-airline fairness measures, which aim to distribute schedule displacement equitably among airlines \cite{fairbrother2018development, jacquillat_roadmap_2018,  zografos_bi-objective_2019, jiang_decision_2021}

These cited imbalances can lead to negative socio-economic and environmental impacts, including increased emissions, and a deterioration of the service quality offered. These consequences affect the travelling public, airlines, airports, and communities situated near airports \cite{keskin_optimal_2023}. Furthermore, \citeonline{zografos2017increasing} suggests that objectives could also be established to maximize social benefits.

Another key motivation, as highlighted by \citeonline{ribeiro_large-scale_2019}, is the limited applicability of exact optimization approaches, which primarily remains confined to small and medium-size airports. This limitation arises from the combinatorial complexity of the problem, rendering it computationally intractable for larger airports. Similarly, \citeonline{zografos2017increasing} supports this view, due to the inherent combinatorial complexity and resource constraints, it is unlikely that a solution algorithm with polynomial time complexity will be found. The complexity of the resulting slot scheduling problem has thus attracted significant interest from the research community, leading to the proposal of several models for optimizing slot allocation decisions \cite{jiang_decision_2021}.

\subsubsection{Concept Mapping: Methods and Results}

To address the research question formulated for this systematic literature review, as detailed in Section \ref{section:Planning}, it is necessary to examine the application of integer programming, queueing theory, and machine learning techniques in addressing the slot allocation problem, consistent with the conceptualization presented in \refFig{fig:narrative_one}. The selected studies have identified three primary methodological branches for solving the allocation problem: (i) single airport allocation and (ii) airport network allocation. A further approach involves (iii) reviewing the existing literature and synthesizing the methodologies and results, which constitutes the objective of the current \acrfull{SLR}. \refFig{fig:narrative_four} illustrates the methodologies employed in the reviewed studies alongside their main findings.

\figuraBib{14d_narrative_section3_method_results_3_R03}{Methodological approaches and main findings of reviewed studies on slot allocation. The articles are referenced by the numbers presented in Table \ref{table*:SRL_narrative_table}}{}{fig:narrative_four}{width=0.8\textwidth}%

\textit{Analysis of existing literature review on slot allocation problem}
\\

Initially, this report will address the secondary studies that provide a literature review.

\citeonline{gillen2016airport}, in their literature review, highlighted the interdependencies among the operating, managerial, and economic dimensions of the slot allocation problem. Their review aimed to identify models that (i) determine appropriate scheduling levels for congestion mitigation and (ii) specify mechanisms for allocating capacity to airlines. The article presented both administrative measures and market-based approaches, explaining and comparing mechanisms such as slot control, congestion pricing, and slot auctions. \citeonline{gillen2016airport} emphasized the inherent uncertainty in addressing delays, meteorological conditions, and the precise timing of movements, which are influenced by various stochastic factors. The authors also explained the complexity of air traffic operations, which can experience multiple bottlenecks across different phases (e.g., departure, en-route, and arrival), each with its own local constraints. \citeonline{gillen2016airport} described models that assess demand performance status but do not focus on optimization. These include deterministic queue models, stochastic queue theory models, and their combinations. To address optimization, the authors cited Integer Programming, which aims to maximize the efficiency of slot allocation. Recognizing the complexity of choosing between price-based and quantity-based instruments, a decision dependent on several market characteristics, the authors established topics to guide the discussion on which method to apply. Finally, \citeonline{gillen2016airport} introduced an integrated framework for airport demand management, outlining the relationships between airport demand, supply, and on-time performance.

\citeonline{zografos2017increasing} aimed to achieve three objectives: (i) a critical review of capacity declaration modeling and optimization strategies for slot allocation within available infrastructure; (ii) an identification of the challenges in existing research and its primary analytical gaps; and (iii) a proposal for future research directions to address the problem. The authors conducted a review of 96 studies, with around two-thirds cited in the article, primarily focusing on slot management approaches rather than market-based methods. The article highlights the necessity of approaches that lead to better quantity (e.g., more slots being accepted than through strict application of \acrshort{IATA} guidelines) and better quality (e.g., allocation closer to the initial submission request). It is demonstrated that a significant relationship exists between slot allocation, runway demand, aircraft fleet assignment, airport gate assignment, landing and takeoff sequencing, crew scheduling, as well as \acrshort{ATM} ground control or delay programming. \citeonline{zografos2017increasing} also addresses the modeling of airport infrastructure capacity declaration, showcasing various assumed attributes (e.g., level of aggregation, types of movements, time intervals for analysis, rolling constraints, capacity influenced by diverse internal and external factors, weather dependency, etc.). The authors not only explain the different types of approaches in terms of infrastructure (e.g., for a single airport, airport network) but also distribution methods (e.g., deterministic or heuristic). To this end, the formulation of an integer linear programming objective function is presented, within a deterministic allocation model, for both infrastructure categories – single and multiple airports. Finally, the article utilizes the unaddressed or explicitly stated needs for future study developments to generate a classification system for the development of new articles. This classification is based on geographical coverage, number of optimization criteria, type of objective function, nature of capacity, types of constraints, precedence relationships, priority classes, and other regulatory properties.

\citeonline{jacquillat_roadmap_2018} outline three primary objectives in their study: (i) the estimation of airport capacity and its key drivers; (ii) the presentation of an approach for determining punctuality performance as a function of demand and capacity; and (iii) the introduction of models that could potentially enhance airport demand management mechanisms, summarizing insights and providing a roadmap for future research. The authors define maximum throughput capacity as the average number of aircraft movements that can be processed per unit of time under continuous demand, identifying its main determinants as the number of runways, runway layout and configuration, separation requirements between movements, operational and meteorological conditions, arrival and departure mix, and aircraft mix. These factors directly influence an airport's ability to receive and accommodate air traffic demand. The authors highlight two ways to characterize airport capacity: theoretical, which utilizes the aforementioned data with certain simplifications to calculate the acceptable demand; and empirical, which, through historical data, estimates the Operational Throughput Envelope to characterize the maximum movements of each category that can be accommodated. Regarding airport congestion, the authors categorize models into three types: microscopic, which creates an airport layout and analyzes each aircraft individually, typically involving simulations; mesoscopic, which characterizes runway utilization, taxi-out and taxi-in times, along with other statistical data, to estimate demand; and macroscopic, which uses an aggregated representation of airport operations to generate accurate estimates of delays based on allocated flights and airport capacity. \citeonline{jacquillat_roadmap_2018} offer insights into the non-linearity between slots, airport capacity, and punctuality performance. They indicate that small increases in slot acceptance can lead to significant negative impacts on punctuality, whereas a more even distribution of flights throughout the day results in lower delay occurrences. Similar to \citeonline{gillen2016airport}, the authors suggest that models such as queueing theory can be applied to understand these sensitivities and, consequently, be utilized with market mechanisms to alter the cost per extra movement during peak hours. In conclusion, the authors define a roadmap of actions following traffic forecast definition: airport infrastructure planning, ensuring the configuration is advantageous by avoiding both under-design (leading to lost demand) and over-design (leading to lost profitability); optimization of air traffic regulation and procedures, which implies training and technology application to meet demand safely, or even enable it; configuration of punctuality performance objectives, determining acceptable levels of congestion; and establishment of demand management mechanisms, whether based on administrative or economic principles.

\hfill \break
\textit{Administrative mechanisms for slot allocation in one airport}
\hfill \break

In addition to the literature review, primary studies can be categorized into approaches for a single airport or an airport network, resulting in 11 main methods: integer linear programming, stochastic queuing, dynamic programming, heuristic solutions, machine learning, multi-objective linear programming, decision framework, game theory model, congestion-aware egalitarian mechanism, mixed-integer programming, and chance-constrained programming.

As demonstrated in the literature review by \citeonline{zografos2017increasing}, a central objective of various approaches is to minimize the deviation between the allocated slot and the airlines' stated need. For approaches applied to a single airport, \citeonline{zeng_data-driven_2021} developed an objective function aimed at minimizing this difference and, consequently, improving punctuality and reducing flight delays. This function minimizes the discrepancy between the allocated slot and the actual predicted arrival time, given a probability of achieving that prediction. Regarding constraints, in addition to runway capacity, the model incorporates airspace corridor capacity, the feasible interval for slot allocation adjustments, the maximum turnaround time, and airport apron capacity. The authors applied the model using data from \acrfull{HGH} during the summer and winter seasons of 2017 – and historical data from the same seasons of the previous year. \acrshort{HGH} is the ninth most congested airport in China by air traffic movements, with two runways and three airspace corridors for arrivals and departures. Compared to the integer programming model by \citeonline{zografos2012dealing} for a single airport – which also considers airport capacity and turnaround time according to IATA guidelines – the model by \citeonline{zeng_data-driven_2021} achieved a reduction in peak departure hours and the average departure time during the day of 80\% and 75\%, respectively. Similar reductions were observed for arrivals, with 60\% and 65\%, respectively. The difference between departure and arrival delays before optimization was 18 minutes; after optimization, it was reduced to 4 minutes. The authors also analyzed the sensitivity of each constraint on the objective function results.

\citeonline{dixit_algorithmic_2023} developed a mechanism termed \acrfull{ECATS}. This method aims to balance several key objectives: efficiency, by allocating slots to those who value them most; egalitarianism, by ensuring fair access for flights connecting remote or less-prioritized cities; and congestion awareness, by preventing over-allocation of flights within specific time slots to mitigate delays and overcrowding. The allocation problem is modeled using an integer programming formulation where the objective function is designed to maximize what the authors term “social utility,” a weighted sum incorporating factors such as the remote city opportunity factor (RCOF) and congestion penalties. This approach facilitates a balance between fairness (egalitarianism) and efficiency. The authors applied the ECATS mechanism at \acrlong{DEL} and \acrlong{MAA}, two airports operating at a coordinated level, over five days in January 2020. While acknowledging their lack of access to precise airport capacity information, necessitating an approximation within the model, the authors anticipated that this would not significantly compromise the experimental results. A payment function, based on the principle of marginal contributions and akin to Vickrey-Clarke-Groves (VCG) payments, is integrated to ensure that each airline pays an amount proportional to its impact on the overall outcome. This payment rule incentivizes truthful bidding and prevents airlines from benefiting through misrepresentation of their valuations. The model's performance was evaluated by comparing its efficiency component against the model proposed by \citeonline{ribeiro2018optimization} and the existing slot distribution within the test sample, demonstrating superior results for \acrfull{ECATS}.

\citeonline{ribeiro2018optimization} presented a multi-objective optimization model, utilizing a linear programming approach, named the \acrfull{PSAM}. The objective function comprises four terms for optimization: the total number of rejected slots, the maximum displacement imposed on any slot, the total displacement across all flights throughout the season, and the total number of displaced slots. Each term is associated with a weight parameter that reflects the priority for optimization, the terms listed above are in their standard priority configuration. The authors also introduced novel constraints that reduce the gap between the integer formulation of the model and its linear relaxation, thereby significantly enhancing its computational performance. This removes the feasible region containing fractional solutions without eliminating any feasible integer solutions, thus ensuring that the optimal integer solution remains unchanged. The article apply the model on the slots requests and allocation data for the Summer Season of 2014 at Madeira and Porto airports in Portugal. It demonstrated exact solutions within reasonable computational times for mid-size airports. Comparisons with slot coordinator decisions indicated that the model accurately captures the main decisions and trade-offs observed in practice. Furthermore, the model improved slot allocation outcomes by reducing the displacement experienced by airlines by an estimated 4.5\% and 27\% at Madeira and Porto airports, respectively. The authors also provided a sensitivity analysis of various constraints imposed by the IATA guidelines.



\citeonline{ribeiro_large-scale_2019} developed a model based on \acrfull{LNS}, a local search heuristic algorithm, to address the airport slot allocation problem. Building upon their previous work \cite{ribeiro2018optimization}, the authors formulated the \acrfull{PSAM-RTA}, increasing the model's complexity to incorporate terminal and apron capacity constraints. Two heuristics were employed: (i) the first generates a good initial feasible solution rapidly by processing slot requests in decreasing order of frequency, and (ii) the second refines the solution through iterative decomposition of the problem into smaller components and subsequent reoptimization. Using data from \acrfull{LIS} in 2015, the authors found that the constructive heuristic provided feasible solutions faster than CPLEX. Furthermore, the outputs of the constructive heuristic, obtained in less than 30 minutes, outperformed the best solutions achieved with CPLEX after 2 days, and the improvement heuristic reduced the optimality gap

\citeonline{kuran_heuristic_nodate} also developed a two-stage heuristic algorithm. The first stage generates a feasible solution by ensuring all capacity constraints are met. The second stage applies a constructive procedure, a heuristic improvement method, focused on assigning requests that could not be accommodated initially to different time slots. The authors applied their model to data from \acrfull{VIE} across five seasons, comparing the results with manually created solutions. The model effectively solved the airport slot allocation problem within short running times, yielding solutions competitive with current operational practices, resulting in low time deviation and minimal fragmentation.

\citeonline{zografos_bi-objective_2019} proposed a bi-objective slot scheduling model aiming to minimize both the total displacement of all slots requested by all airlines and to equalize the average displacement experienced by each airline. The paper offers a comprehensive solution framework employing the $\epsilon$-constraint method, where one objective (fairness) is transformed into a constraint and then iteratively tightened to explore the trade-offs between fairness and efficiency. The presented model was applied to data simulating real-world conditions at a Coordinated Airport. The analysis of the trade-off between the objectives revealed that sacrifices in fairness lead to gains in terms of total displacement. It can be inferred that achieving a fairer scheduling is considerably easier among historic slot requests compared to the scheduling of new entrant and other requests. The efficient frontiers generated by the proposed model can facilitate discussions among various stakeholders, such as airlines, slot coordinators, and airports, to reach a mutually agreed-upon solution. Crucially, these efficient frontiers can also be utilized to avoid dominated (inferior) slot scheduling outcomes.

Building upon the work of \citeonline{zografos_bi-objective_2019}, \citeonline{jiang_decision_2021} introduced three objectives related to fairness: (i) minimizing the maximum deviation from absolute fairness, (ii) minimizing the maximum deviation from average fairness, and (iii) minimizing the Gini Index (MGI), adapted to reflect inter-airline fairness in the slot allocation context. Within this framework, for the MGI, the population under consideration comprises the airlines operating at a given airport, and the characteristic examined is the average displacement allocated to each airline. Selecting the most preferable solution from the generated efficient solutions necessitates the involvement of relevant decision-makers/stakeholders, namely airlines and slot coordinators. In the proposed framework, airlines express their preferences regarding the choice of the efficient frontier, while the slot coordinator makes the final selection using the efficient frontier proposed by the airlines. The authors' approach employs a utility function to identify airlines favored or disfavored by different slot allocation outcomes and a voting mechanism to aggregate airline preferences concerning the efficient frontier to be used for solution selection. Subsequently, the coordinator selects the implemented solution by minimizing the weighted normalized distance from the chosen efficient frontier. The proposed decision-making framework was applied to a setting resembling real-world airport demand and supply conditions. The results of the sensitivity analysis indicate that, irrespective of the fairness objective considered, the number of acceptable slot allocation solutions increases with the importance assigned to fairness. Furthermore, the choice of the most preferable acceptable solution exhibits limited sensitivity to the importance assigned by the slot coordinator to the fairness objective.

\citeonline{jacquillat_interairline_2018} propose and evaluate a quantitative approach to optimize slot distribution, aiming to achieve on-time performance objectives while minimizing interference with airlines’ competitive scheduling and, notably, balancing the impact of such interventions equitably among the airlines. This approach builds upon the \acrfull{ICUSM} framework from \citeonline{jacquillat2015integrated}, incorporating Equity Considerations (\acrshort{ICUSM-E}). The modeling framework of the \acrshort{ICUSM} integrates an Integer Programming model of scheduling interventions, a Stochastic Queueing Model of airport congestion, and a Dynamic Programming model of airport capacity utilization.
The authors introduce the concept of Inter-airline Equity, defined as the ability to balance schedule displacement fairly among airlines. Perfect equity is achieved when the weighted sum of displacements borne by each airline is proportional to the number of flights it has scheduled at the airport. The minimization process is performed lexicographically, first minimizing the largest airline disutility, then the second-largest, and so forth. To achieve this, the authors first establish on-time performance targets, maximizing efficiency, and subsequently optimize equity under these performance and efficiency targets. \citeonline{jacquillat_interairline_2018} applied their model to data from September 18, 2007, at \acrfull{JFK}. Incorporating inter-airline equity objectives in scheduling interventions for flights with similar valuations can yield significant benefits by distributing scheduling adjustments more fairly among airlines without compromising efficiency. Equity can be attained through comparatively small increases in efficiency, even with substantial differences in flight valuations across airlines and within an airline's own flight portfolio.

As demonstrated in the preceding articles, significant effort has been directed towards optimizing slot allocation, with the primary objective of minimizing displacement to airlines’ slot requests. However, slot adherence is susceptible to operational day delays arising from uncertainties such as weather conditions, aircraft maintenance, and passenger-related issues. \citeonline{lambelho_assessing_2020} and  \citeonline{wang_distribution_2022} developed a machine learning approach to predict delays and cancellations at a strategic slot allocation phase (six months prior to operation). The authors employed three machine learning algorithms: \acrfull{MLP}, \acrfull{LGBM}, and \acrfull{RF}. \citeonline{lambelho_assessing_2020} utilized ten strategic flight schedules at \acrfull{LHR} spanning the period from 2013 to 2018. All classifiers achieved an accuracy of 75\% or higher for flight delay classification and 98\% or higher for cancellation classification. Given the imbalanced nature of the training and test data, which contained a larger number of not-delayed/not-cancelled flights compared to delayed/cancelled flights, \acrshort{LGBM} flight classifiers exhibited the best performance in terms of accuracy, precision, and recall \cite{lambelho_assessing_2020}. The authors also presented \acrfull{SHAP} values, illustrating the significant positive or negative impact of specific features on delay/cancellation flight classification and the magnitude of this impact. The features with the highest importance for delay classification were arrival delay, hour of the day, airline type, and aircraft seat capacity. For flight cancellation predictions, the most important features were the origin/destination airport and the operating airline. \citeonline{wang_distribution_2022} tested the algorithms using real data from \acrfull{CAN}, with six flight schedules operated from 2017 to 2020. The results indicated that the accuracy of predicting departure delay at a 65\% confidence level and arrival delay at a 50\% confidence level could exceed 80\%. The authors concluded that \acrshort{MLP} exhibited the worst performance, while \acrshort{LGBM} and \acrshort{RF} showed minimal difference in their results.

\hfill \break
\textit{Administrative mechanisms for slot allocation in network of airports}
\hfill \break

As highlighted in the preceding subsection, the current approaches to address or mitigate the slot allocation problem primarily focus on single-airport scenarios. However, the central problem can also be examined from a broader network perspective, where slot allocation impacts multiple airports \cite{zografos2017increasing}. A primary limitation of single-airport approach lies in its inability to guarantee the usability of an allocated slot for a specific flight. This stems from the fundamental requirement that a flight necessitates a coordinated pair of slots: one at the origin airport and another at the destination airport \cite{wang_slot_2023}. This \acrfull{SLR} identified three main methodological categories applied in the analyzed studies: (i) heuristic algorithms, (ii) mixed-integer programming, and (iii) multi-objective linear programming.

\citeonline{corolli_time_2014} developed two stochastic programming models for the time slot allocation problem, extending the deterministic single-airport model presented in \cite{zografos2012dealing} in two key directions. First, their models simultaneously allocate time slots across multiple airports, explicitly considering – in addition to other operational constraints – the coherence between the departure time slot at the origin airport and the arrival time slot at the destination airport for each flight. Second, their models incorporate stochastic capacity availability, a factor of particular relevance in strategic planning where uncertainty regarding available resources (i.e., capacity) is exceptionally high. Their stochastic optimization approach yields robust solutions by balancing the immediate costs arising from schedule/request discrepancies with the anticipated future costs of delays associated with a proposed flight schedule on operational days. The authors introduced a feature allowing airport declared capacity to be time-dependent, varying period by period on each operational day. This ensures the fullest and most flexible utilization of limited capacity at congested airports. They presented two alternative formulations: the first, termed ‘‘simplified recourse,’’ does not account for the temporal downstream effect of delays, providing a lower-bound estimate of future delays. Its advantage lies in its computability by considering each time instant independently. The second formulation, termed ‘‘time-linked recourse,’’ considers delay propagation between consecutive time instants. The authors evaluated their model on a network of 21 European airports, utilizing four distinct datasets across four calendar days. In one test instance, the time-linked recourse formulation, their most precise, enabled a reduction in total delay costs (the sum of schedule/request discrepancies and operational delays) by over 58\%. Furthermore, the authors noted that their results could also be interpreted as determining the optimal capacity levels at each considered airport that minimize the sum of schedule/request discrepancies and operational delays. The combination of favorable results and computational viability suggests that the proposed approach is highly promising and warrants further investigation, potentially leading to significant monetary benefits for airlines and other stakeholders.


\citeonline{benlic_heuristic_2018} proposed a two-stage heuristic approach. The first stage employs a constructive heuristic procedure to obtain a feasible initial solution by eliminating rotations for which a coherent allocation of slots is not possible. The second stage utilizes an iterative heuristic to optimize the initial solution from the first phase in terms of schedule delay. The authors generated a set of benchmark instances varying in the number of airports within the network and the distribution of requests across these airports. The optimization extends the single-airport slot allocation model presented in \cite{zografos2012dealing}, where the objective is to maximize the number of accommodated requests while minimizing the absolute difference between requested and allocated times. The results indicated that considering the en-route constraint, inherent to the problem, does not introduce a significant degradation in the schedule compared to the existing practice where slots are allocated independently at each airport.

\citeonline{liu_research_2022}  developed a two-step optimization model to investigate the slot allocation problem within an airport network, considering the impact of uncertainty on airport capacity. The first stage of the model operates at an aggregate level, determining flight numbers within a specific timeframe rather than exact slots for each flight, without requiring precise information on airport capacity. Subsequently, at the individual level, each aircraft is assigned to a slot schedule once exact capacity information becomes available. To facilitate more refined management, the authors employed the Benders decomposition algorithm, supported by the Gurobi solver, which enables the specification of multiple scenarios and the construction of a multi-scenario model based on a single base model. The objective is to minimize the total displacement cost and the maximum displacement cost across the defined set of scenarios, without knowing the probability of each scenario. The first stage involves binary integer programming, while the second stage utilizes multi-scenario programming. The Benders decomposition algorithm treats the first-stage integer programming model as the master model, and the second-stage programming is decomposed into sub-models based on the defined scenarios. For their analysis, the authors selected the 15 busiest airports in China, forming an airport network as the analysis system, and used flight requests from March 1 to March 2, 2019. The results indicate that the model's optimization, by considering more information, reduces the number of flight delays and that the conservatism of the two-stage robust optimization is lower than that of single-stage robust optimization.

\citeonline{keskin_optimal_2023} proposed a multi-objective linear programming approach for the simultaneous allocation of slots across a network of airports. The model utilizes the slot allocation outcomes at each individual airport as input for the network-level optimization problem, aligning more closely with the process currently employed by \acrshort{IATA}. This approach enables the consideration of local contexts, as well as the incorporation of primary and secondary criteria in the initial slot allocation at individual airports. It then optimally adjusts the resulting individual airport schedules to ensure network-wide flight connectivity by accounting for the interdependencies between flights connecting pairs of airports. To achieve this, \citeonline{keskin_optimal_2023} developed bi-objective mathematical models incorporating schedule efficiency (Total displacement and Maximum displacement) and fairness objectives, the latter drawing upon the work of \citeonline{jiang_decision_2021}. The authors tested their model on data for coordinated and facilitated airports in Brazil, obtained from the slot coordination database of the Brazilian \acrshort{ANAC}. This dataset included 56 national airports, 16 of which are either facilitated or coordinated, during the summer scheduling season of 2019. The case study results demonstrated that employing connectivity metrics significantly affects the distribution of the total displacement among the airports, although the network-wide total displacement itself remains unaffected. The distribution of displacements across airports varies considerably depending on the connectivity metric used. Their analysis regarding inter-flight fairness suggests a weak trade-off between total displacement and inter-flight fairness. This implies that a fairer network-wide schedule can be generated without significantly compromising the efficiency of the allocation.

\citeonline{pellegrini_sosta_2017} developed the \acrfull{SOSTA} model, a decision support tool designed for the optimal coordination of airport capacity management at the European level. \acrshort{SOSTA} aims to optimize slot allocation by (i) penalizing the non-allocation of requested slots, (ii) penalizing the displacement of requests not belonging to a pair of coupled requested slots, and (iii) penalizing the displacement of the requested departure slot and the block time difference resulting from the allocation. The model was tested at Level 2 and Level 3 European airports on June 28, 2013, the day with the highest number of movements in that year. The authors highlighted the difficulty in accessing data on requested airport slots and noted that the available real-world data presented only a small percentage of flight records with discrepancies between requested and allocated slots - this was likely attributed to coordinators having already optimized their flights post-allocation. The results indicated that, across all cases, the proposed displacement differed from the actual displacement by a maximum of 10 minutes. A slightly different combination enabled a reduction in the total displacement by 5 minutes. The final slot allocation led to the saturation of some interval capacity constraints at several airports. In summary, the slot allocation generated by \acrshort{SOSTA} closely aligns with the outcome of the actual slot allocation process, suggesting that it effectively reproduces IATA slot regulations and best practices. \citeonline{pellegrini_sosta_2017} also conducted a sensitivity analysis of demand and capacity imbalances and introduced a fairness term for the method.

To allocate airport slots within a \acrfull{MAS}, \citeonline{wang_slot_2023} proposed a novel uncertainty slot allocation model specifically for an \acrshort{MAS}. This model considers fixed capacity constraints alongside the uncertainty associated with flying times between airports and fixes. This uncertainty is formulated using chance constraints, where a violation probability is set as an upper bound, limiting the likelihood of constraint violations. To transform these chance constraints, the authors employed stochastic programming to derive a deterministic model transformation. A scenario generation model, as detailed in \citeonline{liu_research_2022}, is developed to determine flying times and the corresponding probability of each potential scenario. The objective function aims to minimize the total displacement of all flights, thereby reducing the cost of slot interventions, and is formulated as a mixed-integer programming model. The authors applied their model to a multi-airport system in the Guangdong-Hong Kong-Macao Greater Bay Area (GBA), recognized as one of the world's busiest \acrshort{MAS}. This area studied comprises three Level 3 airports: \acrfull{CAN} and \acrfull{SZX}; two Level 2 airports: \acrfull{ZUH} and \acrfull{MFM}; and one Level 1 airport: \acrfull{HUZ}. They utilized the actual flight schedule for a single day (December 21, 2019) as input for their model. The results revealed a negative correlation between the changing rate of total displacements and the violation probability. A smaller violation probability implies stricter constraints, leading to significant changes in allocation results even with small variations in violation probability. Conversely, a larger violation probability entails the consideration of more scenarios, making the total displacements more sensitive to changes in this probability. The results also indicated that displacements were not solely concentrated at the airport with the highest traffic volume. These differences can be partly attributed to the fact that displacements are influenced by both demand and capacity. Furthermore, the study noted that displacements were not evenly distributed across arrival and departure flights. In a deterministic model, while schedules are optimized to meet capacity, there remains a high probability of airspace congestion. In contrast, by considering flying time uncertainty, the likelihood of violating capacity constraints is reduced. Thus, the authors concluded that the robustness of the optimized flight schedule can be substantially improved when allocating airport slots by incorporating flying time uncertainty.

\subsubsection{Concept Mapping: Limitations}

The articles selected for this \acrfull{SLR} limitations that can be categorized as follows: (i) Research Gaps, these often stem from assumptions or methodological decisions made during the studies, which, while enabling the pursuit of their research questions, introduce certain caveats; and (ii) Future Work,  this encompasses opportunities to expand the research topic and its scope, or to explore subjects not fully addressed in the current articles due to divergence from their primary objectives or a potential for improvement if combined with alternative approaches.
\refFig{fig:narrative_five} outlines the primary limitations within each category, emphasizing the relationships among the articles pertaining to this topic.


\figuraBib{14c_narrative_section3_limitation_2.drawio}{Research gaps and future directions: a synthesis of the selected studies limitations. The articles are referenced by the numbers presented in Table \ref{table*:SRL_narrative_table}}{}{fig:narrative_five}{width=0.7\textwidth}%

\hfill \break
\textit{Research Gaps}
\hfill \break

One prominent research gap identified is the necessity of addressing the complexity of real operational costs. Studies frequently simplify the number of factors analyzed, often due to inherent methodological choices, which directly impacts their results \cite{wang_distribution_2022}. For instance, \citeonline{zografos_bi-objective_2019} acknowledged their premise that economic costs associated with flight displacement were uniform across all flights of the same airline and did not vary between airlines. This simplification may not hold true given diverse business models. This issue could potentially be resolved by incorporating airline preferences regarding the allocation of their schedule displacement to various flights at different airports \cite{keskin_optimal_2023, pellegrini_sosta_2017}. \citeonline{corolli_time_2014} further emphasized the need to compare costs of schedule definition in strategic and operational phases, as the former incurs additional crew and passenger-related costs, leading to higher overall expenditures than strategic planning.

A recurring concern in the literature is the challenge of accurately representing real-world airport operational capacity. \citeonline{ribeiro2018optimization} emphasized the necessity of capturing additional complexities from IATA guidelines, such as terminal, apron, and noise restrictions, which they subsequently addressed in an extended version of their model \cite{ribeiro_large-scale_2019}. \citeonline{zografos2017increasing}, in their integrated review, elucidated the intricacy of the declared capacity construct. This construct encompasses various attributes, including distributions of arrivals and departures, the number of available slots per time interval, the unit of time, rolling constraints, airport infrastructure, and weather dependency. Achieving exact solutions while increasing capacity constraints to mirror real-world problems, by accounting for the aforementioned characteristics, can substantially increase computational processing time. This makes such approach less favorable for practical application \cite{benlic_heuristic_2018, wang_slot_2023}.

Other identified gaps include the adaptation to missing initial slot request data \cite{pellegrini_sosta_2017, keskin_optimal_2023}, as well as the computational performance required for models to provide exact solutions \cite{ribeiro2018optimization}.

\hfill \break
\textit{Future work}
\hfill \break

The studies selected for this systematic literature review (\acrshort{SLR}) consistently include suggestions for future work based on their methods and results. This review has identified eight primary recommendations for further research directions. Some of these have already been addressed in subsequent studies, following the guidance of previous research, and were reviewed in the preceding sections.

The most frequently cited area for future work is the development of heuristic solution algorithms. This is particularly relevant for addressing similar problems at even larger schedule-coordinated airports, as such algorithms could significantly reduce computational costs \cite{ribeiro2018optimization, zografos_bi-objective_2019, liu_research_2022}. \citeonline{ribeiro_large-scale_2019} expanded upon this suggestion by creating a method that combines approximations with exact solutions. \citeonline{keskin_optimal_2023} also identified heuristics as a viable alternative for approaching the slot allocation problem within airport networks.

As previously noted, the methodologies for slot allocation can be categorized into approaches for single airports or airport networks. Methods developed for the former could potentially be expanded to address the latter. By integrating various elements, including sector capacity, route priority, and delay propagation effects, the concept of a data-driven model for single airport can also be extended to airport networks \cite{corolli_time_2014, ribeiro_large-scale_2019, zeng_data-driven_2021}. \citeonline{benlic_heuristic_2018}, whose research addressed the slot allocation problem through a heuristic algorithm in an airport network context, recommend that network-level approaches should ensure robust algorithmic performance in instances more closely resembling real-world cases, and generalize findings regarding the implications of alternative slot-scheduling policies.

Regarding equity issues in slot distribution, the potential benefits of scheduling interventions also motivate future research directions. There is a recognized need to implement methods that effectively balance fairness and efficiency \cite{wang_slot_2023}. A significant opportunity exists in the design and optimization of scheduling intervention mechanisms that allow airlines to provide their preferred flight schedules \cite{jacquillat_interairline_2018, jiang_decision_2021, keskin_optimal_2023}. The information shared through such mechanisms should be implemented in a manner that prevents "gaming" by airlines misrepresenting their true information \cite{corolli_time_2014}. \citeonline{zografos_bi-objective_2019} emphasized the necessity of comparing alternative fairness measures proposed in the literature and developing a slot allocation mechanism that incorporates both efficiency and fairness criteria. Another opportunity involves investigating the inclusion of fairness metrics that differentiate between peak and off-peak slot requests, or fairness measures that integrate schedule displacement costs \cite{jiang_decision_2021}. \citeonline{pellegrini_sosta_2017} noted that the current slot allocation process, established by \acrshort{IATA}, faces criticism for its perceived lack of fairness, barriers to entry (attributed to grandfather rights), and inefficient use of airport capacity, and future work should explore regulation alternatives.

For multi-objective functions, a key area for future work involves the appropriate definition and study of weight factors \cite{corolli_time_2014}. In the longer term, models could address strategic questions related to airport declared capacities, the definition and prioritization of slot allocation objectives, and mechanisms for airport capacity allocation. This would involve accounting for their individual or joint effects on airline schedules, airport operations, and passenger demand. Investigating the efficient trade-off frontier between the objectives considered represents an important avenue for future research \cite{ribeiro2018optimization}.

Additional limitations noted for further research and exploration include: demand forecasting, the application period for slot distribution in airport networks, the application of machine learning algorithms for data prediction, fragmentation within the slot allocation system, and the trade-offs between safety and efficiency.

A primary motivation for studying the slot allocation problem stems from the increase in air traffic not being matched by corresponding infrastructure availability. A significant challenge in this context is the forecasting of demand over typical time horizons of thirty to forty years. Consequently, it is imperative that such demand forecasts be developed for a variety of scenarios, reflecting the high variability inherent in the regulatory, competitive, and general economic environment of air transport \cite{jacquillat_interairline_2018}.

\citeonline{pellegrini_sosta_2017} applied a method to an airport network that incurred high computational costs. They noted that considering only a single day for slot allocation is limiting, given that the slot allocation process prioritizes the assignment of slot series. The authors suggested that the slot allocation problem for an entire season in an airport network, such as in Europe, could be solved heuristically, beginning with the busiest days of the season.

For machine learning models, which utilize historical data to categorize future events, it is believed that prediction performance could be enhanced through the construction of more sophisticated models or the development of more precise loss functions tailored to each algorithm used for the slot allocation problem \cite{zografos_bi-objective_2019}. \citeonline{lambelho_assessing_2020} suggested extending the feature set for prediction algorithms to improve accuracy. Future work in this area will evaluate the impact of incorporating flight delay and cancellation predictions into flight scheduling optimization models at the strategic phase.

Fragmentation, defined as the distribution of a set of related slot requests (often termed a series) across different time slots rather than their ideal grouping, essentially quantifies how "spread out" assigned times are for requests that ideally should share a common time slot. This concept applies to airport slot allocation, where the goal is to maintain a consistent and uniform schedule for flights during a season. High fragmentation can lead to delays and inefficiencies. Given its inherent presence in airport operations, the development of strategies to mitigate delays due to fragmentation should continue to be explored \cite{kuran_heuristic_nodate}.

\citeonline{liu_research_2022} urged the research community to address its increasing focus on efficiency as the sole objective function criterion, often treating safety capacity merely as a constraint. Consequently, they recommended a more detailed investigation into the trade-off between security and efficiency as dual objectives.


\subsection{Assessing the robustness of the synthesis: Critical Reflection}


Em construção