\section{Planning the review}
\label{section:Planning}

To gain a comprehensive understanding of the main problem and existing approaches, a \acrfull{SLR} was conducted following a transparent process to minimize bias and enhance reliability.

The \acrshort{SLR} adhered to a three-step methodology: (i) Planning, this stage involved defining the research need, translating research questions into specific search terms, and developing a detailed review protocol prior to data collection; (ii) Conducting the Review, this stage encompassed identifying relevant research addressing the slot allocation problem, selecting primary studies, assessing their quality, extracting key data, and synthesizing the findings; and (iii) Presenting the Review Results, this stage aimed to summarize the key research findings related to the central problem and demonstrate how the proposed methods in this paper can address identified gaps in the existing literature.

A thorough understanding of airport slot allocation necessitates a comprehensive literature review to identify, analyze, and interpret diverse perspectives on the problem and its existing solutions. This review provides historical context and clarifies the potential contributions of this paper. To ensure a bias-mitigated and transparent data collection process, the review criteria and procedures were meticulously planned before execution, adhering to the guidelines outlined in \citeonline{wohlin2012experimentation} and \citeonline{shamseer2015prisma}. \refFig{fig:experiment} illustrates the sequential processes and their respective deliverables.

\figuraBib{10_literaturaReview_en}{The literature review processes and their deliverables. Adapted from: \cite{wohlin2012experimentation}}{}{fig:experiment}{width=0.6\textwidth}%

\subsection{The need for a review}

The first step was the identification of the need for a review, i.e., understand how the problem has been addressed in past research and what are the previous limitation. 

Airport infrastructure exhibits significant variability, encompassing distinct configurations of runways, taxiways, gates, and terminals. Furthermore, airports operate within diverse environments, including densely populated areas, which significantly influences demand. These unique characteristics can lead to diverse perspectives on the slot allocation problem and consequently, a range of proposed solutions. Given the well-established nature of the slot allocation problem, a thorough understanding of its historical management is crucial.


While a \textit{snowballing} approach could potentially gather relevant literature, it carries the risk of introducing search bias, potentially limiting the comprehensiveness and reproducibility of the review \cite{jalali2012systematic}. A systematic review offers a more rigorous approach, ensuring a comprehensive understanding of the problem through a well-defined, step-by-step protocol that is subject to review and potential improvement.

This review assumed the absence of prior systematic reviews on this specific topic. However, if any such reviews were encountered, their contributions and limitations would be carefully evaluated

\subsection{The \acrshort{SLR} researches questions}

The second step involved formulating specific research questions to guide the literature search. These research questions serve as crucial focal points for identifying relevant primary studies and extracting and analyzing relevant information. To ensure clarity and focus, the PICO framework was employed to systematically break down each research question into its essential components: Population, Intervention, Comparison, and Outcome.

Furthermore, the 'Intervention' component was specifically defined to clarify the approach of the present paper and determine whether similar approaches have been previously utilized in the literature.

The following research questions were formulated for this literature review:

\begin{itemize}
    \item \textbf{RQ1:} In medium and large airports (P), how do integer programming, queue theory, and machine learning techniques (I)  compared to the \acrshort{IATA}'s guidelines for the administrative management of slots (C) in terms of their effectiveness in reducing the number of changes to airlines' initial allocations and optimizing airport infrastructure utilization (O)?
\end{itemize}

To further refine the inquiry, the following three sub-questions were formulated: (i) What methodologies have been employed to solve the airport slot allocation problem? (ii) What are the primary constraints associated with these approaches? (iii) Are there any identified gaps or underlying assumptions that warrant further exploration? It aims to collect the most valuable information about the models created to solve the problem and how experimentation with algorithms are helping them to tackle it.

\subsection{Review protocol}

The third and final step in the planning phase involved the development of a comprehensive review protocol. This protocol serves as the foundation for the current research review and plays a crucial role in ensuring the reproducibility and validity of the study. The following key elements were considered during the development of the review protocol:

\subsubsection{Search strategy for primary studies}

The primary activities within this step involved:

\begin{itemize}
    \item \textbf{Formulating a comprehensive search string:} this was achieved by translating the research questions into a set of relevant keywords.
    \item \textbf{Executing database searches:} the formulated search string was then applied to a range of relevant academic databases.
\end{itemize}

The search string was constructed using a combination of keywords, including: "Airport", "Slot", "Allocation", "Integer Programming", "Queue Theory", "Machine Learning", "Optimization", and "IATA Slot Guidelines".

To refine the search, boolean operators were employed to effectively combine these keywords, resulting in the following search string:
\\

\textit{(airport* \textbf{AND} "slot allocation" \textbf{AND}
 optimization \textbf{AND} ("IATA guidelines" \textbf{OR} "Worldwide Airport Slot Guidelines") \textbf{AND} ("machine learning" \textbf{OR} "integer programming" \textbf{OR} "queue theory")) \textbf{AND} [E-Publication Date: (01/01/2002 \textbf{TO} 01/12/2023)]}. 
\\

\subsubsection{Search sources}

The formulated search string was subsequently employed to systematically search the following academic databases to identify relevant primary research studies:
\begin{enumerate}
    \item ACM Digital Library;
    \item \acrfull{BASE};
    \item EBSCO;
    \item ScienceDirect;
    \item Scopus;
    \item \acrfull{TRID};
    \item Web of Science;
    \item \acrfull{DBLP};
    \item \acrfull{DOAJ};
    \item IEEE Xplore;
    \item JSTOR; and
    \item Springer Link (Article section).
\end{enumerate}


The selection of databases was guided by two primary criteria: (i) Multidisciplinary Coverage, the databases were chosen to encompass both multidisciplinary subjects and those specializing in computer science or transportation studies, aligning with the core research areas of this paper; and (ii)  Search Capability, the databases were selected based on their comprehensive coverage and robust search capabilities, as outlined in \citeonline{kitchenham2007guidelines}, this selection criterion prioritizes databases that facilitate systematic searches through the utilization of queries and filters, ultimately ensuring the reproducibility of the search process with high recall and precision.  

Following the criteria established by \citeonline{gusenbauer2020academic}, databases were classified as 'primary' if they met all necessary quality requirements, including the possibility of query refinement, string size criteria, server response criteria, use of boolean operators, functionalities and reproducibility. Databases 1 to 7 were designated as primary databases, while the remaining databases were considered supplementary. No additional search strategies (e.g., searching specific journals and conference proceedings or contacting researchers to obtain unpublished material or grey literature) were employed.

This step necessitates the meticulous removal of all identifiable false positives and duplicate entries across the searched databases.

To validate the search string's effectiveness, the article by \citeonline{ribeiro2018optimization} was used as a benchmark. This article, previously identified as relevant to the research scope, must be retrieved within the search results from at least one of the targeted databases. Failure to retrieve this article necessitates a reformulation of the search string.

\subsubsection{Study selection criteria}
To mitigate bias, specific inclusion and exclusion criteria were established prior to the screening process. Employing the "First Pass" approach as established by \citeonline{keshav2007read}, all primary papers were screened by carefully reviewing their titles, abstracts, introductions, and conclusions

\noindent \textit{Inclusion criteria:}

\begin{enumerate}

    \item The study must address any aspect of the airport slot allocation problem, encompassing models, perspectives, constraints, methods, technical considerations, etc.; and
    \item The study must propose a solution for the airport slot allocation problem utilizing integer programming, queue theory, or machine learning techniques.
\end{enumerate}

\subsubsection{Study exclusion criteria}

\textit{Exclusion criteria:}
\begin{enumerate}
    \item \textbf{Duplicates and Old Versions}, articles are excluded if they are duplicates of other included articles or represent older versions of the same study. If an article appears in multiple primary and/or secondary databases, it is retained in the database listed first in the "Search Sources" section and removed from the remaining databases.
    \item \textbf{Publication Date}, articles published before 2001 are excluded. This timeframe is chosen to focus on research conducted after the significant market regulations implemented following the September 11th attacks.
    \item \textbf{Scope of Slot Guidelines}, articles introducing slot allocation guidelines beyond those outlined in the \acrshort{IATA}'s Worldwide Slot Guidelines (\acrshort{WASG}) \cite{WASG2020} are excluded.
    \item \textbf{Focus on Auction-based Solutions}, articles that primarily focus on auction-based economic approaches to solving the airport slot allocation problem are excluded.
    \item \textbf{Lack of Full-Text Access}, articles that are not accessible in full-text format are excluded.
    \item \textbf{Irrelevant Scope}, articles that are not directly relevant to the research questions and do not address the core slot allocation problem, but instead focus on adjacent areas such as ground delays or weather impacts on airports, are excluded.
    \item \textbf{Specific Context}, articles that focus on specific contexts, such as the September 11th attacks or the COVID-19 pandemic, are excluded to maintain a broader and more generalizable perspective.
    \item \textbf{Language}, articles not published in English or Portuguese are excluded.
\end{enumerate}

\subsubsection{Study selection procedures}

The removal of all irrelevant papers was conducted through a thorough assessment of each study against the established selection and exclusion criteria. This process involved a careful review of titles, abstracts, and conclusions, with a focus on identifying and excluding papers that did not meet the inclusion criteria. If more than one section of the paper (title, abstract, or conclusion) was deemed irrelevant to the scope of this review, the paper was excluded, and a concise justification for its exclusion was recorded.

\subsubsection{Narrative Synthesis}

 Given the anticipated variability and potential differences among the included studies – stemming from the application of diverse methodologies across varied samples (e.g., different airports) and potentially leading to methodological heterogeneity - this review will employ a narrative synthesis approach to synthesize the extracted data and assess the overall body of evidence \cite{campbell2018improving, campbell2019lack}. 
 
 This method is particularly suitable when the heterogeneity of the included studies precludes statistical meta-analysis \cite{popay2006guidance, campbell2018improving, campbell2019lack}. 
 
 Drawing upon the guidance for Narrative Synthesis outlined by \citeonline{popay2006guidance}, this study will undertake the following key stages: 
 \begin{enumerate}
     \item \textbf{Developing a theoretical understanding of the problem:} This stage will involve constructing an initial framework or theory of how the identified software problem is conceptualized and addressed within the existing literature, considering the different approaches and contexts. 
     \item \textbf{Developing a preliminary synthesis of findings through tabulation:} A systematic tabulation of key characteristics and findings from each included study will be created. This will facilitate an initial overview and comparison of the evidence.
     \item \textbf{Exploring relationships within the data using concept mapping:} Concept mapping techniques will be employed to visually represent and explore the relationships between different concepts, interventions, outcomes, and contextual factors identified across the studies. This will aid in identifying patterns and potential associations. 
     \item \textbf{Assessing the robustness of the synthesis through critical reflection:} The final stage will involve a critical discussion of the synthesis process, considering the strengths and limitations of the included studies (as determined by the quality assessment), the potential for bias, and the overall confidence in the synthesized findings. This reflection will explicitly address the impact of methodological heterogeneity on the conclusions drawn.
 \end{enumerate}

\section{Conducting the review}
\label{section:Conducting}
This section presents the outcomes of the systematic search and selection process, including a comprehensive review of the validity checks implemented throughout the study. These checks encompassed data extraction procedures and a rigorous quality assessment of the included studies.

\subsection{Identification of research}

The initial search yielded 31 sources across 12 databases. After removing 9 duplicates (29.03\%), a total of 22 unique sources remained for further analysis, as detailed in Table \ref{table*:results}.

\begin{table*}[h]
\begin{center}
\caption{Search results per database}
\small
\begin{tabular}{*{15}{l}}
\hline
\multicolumn{1}{l}{\#} & \textbf{DATABASE}                                     & \multicolumn{1}{l}{\textbf{SOURCES FOUND}} \\ \hline
1  & ACM Digital Library                      & 20 \\
2  & Bielefeld Academic Search Engine (BASE)  & 8  \\
3  & EBSCO                                    & 20 \\
4  & ScienceDirect                            & 74 \\
5  & Scopus                                   & 8  \\
6                      & Transport Research International Documentation (TRID) & 8                                          \\
7  & Web of Science                           & 6  \\
8  & CiteSeerX                                & 0  \\
9                      & Digital Bibliography \& Library Project (DBLP)        & 8                                          \\
10 & Directory of Open Access Journals (DOAJ) & 3  \\
11 & IEEE Xplore                              & 1  \\
12 & JSTOR                                    & 1  \\
13 & Springer Link                                         & 19                     \\ \hline
   & \textbf{TOTAL}                                        & 146                    \\ \hline
\end{tabular}
\label{table*:results}
\end{center}
\end{table*} 

All studies were identified within the designated primary databases, with ScienceDirect emerging as the most prolific source, contributing 59.09\% of the total retrieved articles. Notably, three databases – \acrshort{DBLP}, \acrshort{DOAJ}, and IEEE Xplore – did not yield any results in the initial search, even when considering duplicates.

To ensure a comprehensive understanding of the relevant literature, a rigorous screening process was implemented. Following the "first pass" approach outlined by \citeonline{wohlin2012experimentation}, each study's title and abstract were independently and randomly reviewed. Studies were excluded if they did not directly address the airport slot allocation problem or failed to meet the established inclusion criteria. This screening process resulted in the selection of 19 primary studies, representing 86.36\% of the unique sources identified.
\\

For the exclusion criteria: 
\begin{enumerate}
    \item One article was not accessible;
    \item One article was reviewing the slot allocation problem through economic approaches; and
    \item One article was focusing in adjacent area (flight planning optimization).
\end{enumerate}

Following the selection procedures, the systematic literature review (SLR) yielded 19 primary studies. Notably, 13.63\% of the initial search results were identified as false positives, i.e., studies retrieved by the search string but ultimately deemed irrelevant to the research scope. Table \ref{table*:falsepositive} provides a summary of the primary studies identified per database, including the number of false positives observed after duplicate removal.


\begin{center}
    
\setlength{\tabcolsep}{10pt} % Default value: 6pt
\renewcommand{\arraystretch}{1.5} % Default value: 1
\begin{xltabular}{\textwidth}{p{0.5cm} p{7cm} p{3cm} p{2.6cm}}
\caption{Primary studies per database and the false positive results.} \label{table*:falsepositive} \\

\hline \multicolumn{1}{c}{\#} & \textbf{DATABASE}                                     & \textbf{PRIMARY STUDIES} & \textbf{FALSE POSITIVES} \\ \hline 
\endfirsthead

\multicolumn{4}{c}%
{\tablename\ \thetable{} - Continued from previous page.} \\
\hline \multicolumn{1}{c}{\#} & \textbf{DATABASE}                                     & \textbf{PRIMARY STUDIES} & \textbf{FALSE POSITIVES} \\ \hline 
\endhead

\hline \multicolumn{4}{r}{{Continued on next page.}} \\ \hline
\endfoot

\hline
\endlastfoot

1                    & ACM Digital Library                            & 4  & 80.00\%     \\
2                    & Bielefeld Academic Search Engine (BASE)        & 4  & 50.00\%  \\
3                    & EBSCO                                          & 3  & 83.33\%  \\
4                    & ScienceDirect                                  & 30 & 58.33\%  \\
5                    & Scopus                                         & 0  & -   \\
6                      & Transport Research International Documentation (TRID) & 2                                            & 0.00\%                   \\
7                    & Web of Science                                 & 0  & -        \\
8                    & CiteSeerX                                      & 0  & -        \\
9                    & Digital Bibliography \& Library Project (DBLP) & 1  & 66.67\%  \\
10                   & Directory of Open Access Journals (DOAJ)       & 1  & 33.33\%  \\
11                   & IEEE Xplore                                    & 0  & -        \\
12                   & JSTOR                                          & 0  & - \\
13                   & Springer Link                                  & 2  & 88.89\%  \\ \hline
\multicolumn{1}{l}{} & \textbf{TOTAL}                                          & 47 & 67.81\%  \\ \hline
\end{xltabular}

\end{center}

The distribution of studies by year of publication is visualized in \refFig{fig:yearPublication}. Notably, the subject of this investigation has gained significant attention in recent years, with 43.37\% of the selected sources published between 2020 and 2023. The selection process is illustrated in \refFig{fig:resumeSLR}.

\figuraBib{12a_yearPublication}{Number of studies by year of their publication}{}{fig:yearPublication}{width=0.8\textwidth}%

\figuraBib{11a_SLR_brief}{The studies selection procedures and deliverable. Adapted from: \cite{page2021prisma}}{}{fig:resumeSLR}{width=1.0\textwidth}%


\section{Review report process}

\subsection{Theoretical understanding}
Initially, it is feasible to approach the review by synthesizing multiple pieces of evidence extracted from the individual studies included, in order to construct a model that highlights key concepts or issues pertinent to the review question and delineates the relationships among them \cite{popay2006guidance}.

The air transport industry, encompassing airlines, airports, and local regulatory agencies, has sought methods to utilize slot allocation coordination as a means to optimize the available airport infrastructure. This approach presents a viable solution to address the rapid increase in air transport demand in the short term. Consequently, two primary questions arise: (i) how can slot distribution be optimized to align with the airport's available infrastructure, and (ii) what are the constituent elements of the airport's operational capacity?

Stakeholders are pursuing a model that facilitates the efficient utilization of airport infrastructure, reduces delays, enhances the probability of airlines retaining allocated slots, and ensures the necessary equity in the distribution of these scarce resources. The long-term impact of such a model includes a more comprehensive understanding of airport limitations, as well as the improvement of the financial performance of all stakeholders and the current slot regulations. 

The figure \refFig{fig:narrative_one}.provided illustrates the theoretical framework of the problem, progressing from the desired impact and its associated benefits and metrics, to the principal activities required to initiate the model for a more effective resolution of the initial problem (theory of change).

The validity of the initial understanding will be demonstrated within the following sections of the report. The aim is to answer the research questions posed by the systematic literature review

\figuraBib{13_narrative_section1}{Theoretical understanding of the problem: Linking slot optimization and airport sensitivity to enhanced allocation and financial outcomes. Adapted from: \cite{popay2006guidance, eduspots_theoryofchange}}{}{fig:narrative_one}{width=1.0\textwidth}%


\subsection{Preliminary synthesis of findings through tabulation}

To synthesize the information derived from the 19 selected studies, the Table \ref{table*:SRL_narrative_table} was constructed. The objective of this table was to systematically present each study, detailing its title, applied methodologies, population under investigation, and principal findings. The numerical designation assigned to each article will serve as a consistent reference throughout this report.

Within the table, in addition to the primary methodologies employed in each study, the applicability of each model was determined based on the coverage classification—namely, whether it pertained to a single airport or an airport network \cite{zografos2017increasing}. Furthermore, to facilitate a more comprehensive understanding of the approaches and to underscore the heterogeneity in models, data, and results across the cases, the specific population of each study is also indicated (e.g., the name of the airport or the number of airports comprising the airline network). The main results of each study were explicitly stated in the table, albeit not exhaustively, encompassing both implicit and explicit outcomes and contributions of the respective models.

\clearpage

\begin{sidewaystable}

\caption{Primary studies collection} \label{table*:SRL_narrative_table}
\tiny
%\begin{tabular}{llllll}
\begin{tabular}[H]{p{0.25cm} p{3cm} p{5cm} p{3.5cm} p{3.5cm} p{8cm}}
\toprule
\centering

\textbf{\#} & \textbf{CITATION} & \textbf{TITLE} & \textbf{METHOD} & \textbf{POPULATION} & \textbf{RESULT} \\

\midrule

1 & \citeonline{zeng_data-driven_2021} - Research paper & A data-driven flight schedule optimization model considering the uncertainty of operational displacement &  Linear programming method (Single airport) & \acrlong{HGH} (two season) & This research demonstrates a reduction of over 60\% in both arrival and departure traffic compared to other optimization model. \\
\midrule
2 & \citeonline{ribeiro_large-scale_2019} - Research paper & A large-scale neighborhood search approach to airport slot allocation & Heuristic solution method and integer programming method (Single airport) & \acrlong{LIS} (two seasons) & The authors' algorithm produced solutions within 0.1\% of the optimum in just a few hours. \\
\midrule
3 & \citeonline{zografos2017increasing} - Integrated review & Increasing airport capacity utilisation through optimum slot scheduling: review of current developments and identification of future needs & Literature review & N/A &  This review encompassed 96 research papers published between 1993 and 2014, developing a classification system for the approaches used in the slot allocation problem.\\
\midrule
4 & \citeonline{ribeiro2018optimization} - Research paper & An optimization approach for airport slot allocation under \acrshort{IATA} guidelines & Integer programming method (Single airport)   & Madeira and Porto airports (one season)  & The model improves slot allocation outcomes by reducing the displacement experienced by airlines by an estimated 4.5\% at Madeira and 27\% at Porto. \\
\midrule
5 & \citeonline{benlic_heuristic_2018} - Research paper & Heuristic search for allocation of slots at network level & Heuristic solution method (Network of airports) & Set of benchmark (half year) & The article demonstrates that even under strict rules, a network-wide allocation method can achieve a coherent schedule that minimizes delays. \\
\midrule
6 & \citeonline{wang_distribution_2022} - Research paper & Distribution prediction of strategic flight delays via machine learning methods & Machine learning-based model (Single airport) & \acrlong{CAN} (3 years) & Prediction accuracy exceeding 0.80 was achieved. \\
\midrule
7 & \citeonline{jacquillat_interairline_2018} - Research paper & Interairline equity in airport scheduling interventions &  Lexicographic modeling (Single airport) & \acrlong{JFK} (one calendar day) & The article shows that achieving maximum equity requires no (or minimal) sacrifice in terms of efficiency losses. \\
\midrule
8 & \citeonline{zografos_bi-objective_2019} & A bi-objective efficiency-fairness model for scheduling slots at congested airports & Bi-objective modelling framework (Single Airport) &  Data resembling real world conditions of a Coordinated Airport (one season) & The paper presents a bi-objective model that combines efficiency and fairness. The model uses the $\epsilon$-constraint method, where one objective (fairness) is turned into a constraint and then iteratively tightened to explore trade-offs between fairness and efficiency.  \\
\midrule
9 & \citeonline{jiang_decision_2021} & A decision making framework for incorporating fairness in allocating slots at capacity-constrained airports &  Bi-objective modelling framework (Single Airport) & One airport (one season) & The paper presents a framework that combines efficiency and fairness. A key innovation is the incorporation of a voting mechanism. Airlines express their willingness to trade-off efficiency for fairness depending on their individual weight. \\
\midrule
10 & \citeonline{jacquillat_roadmap_2018} - Integrated review  & A roadmap toward airport demand and capacity management & Literature review & N/A & The review provides a framework that underscores the critical interdependencies between operational/managerial, and economic considerations in airport demand management. \\
\midrule
11 & \citeonline{gillen2016airport} - Integrated review & Airport demand management: The operations research and economics perspectives and potential synergies] & Literature review & N/A & Introduced an integrated framework and highlighted potential research synergies at the intersection of operations, economics, and management. \\
\midrule
12 & \citeonline{dixit_algorithmic_2023} - Research paper & Algorithmic mechanism design for egalitarian and congestion-aware airport slot allocation & A game-theoretic model and a mechanism design solution (Single airport) & \acrlong{DEL} and \acrlong{MAA} (5 calendar days) & The model yielded a 5-20\% and 20-30\% increase in social utility at \acrshort{DEL} and \acrshort{MAA}, respectively. \\
\midrule
13 & \citeonline{kuran_heuristic_nodate} - Thesis & Heuristic optimization methods for seasonal airport slot allocation &  Heuristic solution method (Single airport) & Vienna Airport (5 seasons) & The proposed method demonstrates promising results in minimizing time deviations from the requested slots. \\
\midrule
14 & \citeonline{lambelho_assessing_2020}  - Research paper & Assessing strategic flight schedules at an airport using  machine learning-based flight delay and cancellation predictions & Machine learning-based model (Single airport) & \acrlong{LHR} (10 seasons) & The proposed prediction models demonstrated high accuracy (0.79) across all evaluated approaches. \acrfull{LGBM} consistently exhibited superior performance. \\
\midrule
15 & \citeonline{keskin_optimal_2023} - Research paper & Optimal network-wide adjustments of initial airport slot allocations with connectivity and fairness objectives & Two bi-objective models (Network of airports)  & 16 coordinated and facilitated airports in Brazil (2019 summer season) & The case study results indicate that using connectivity metrics significantly affects the distribution of total displacement among airports. \\
\midrule
16 & \citeonline{liu_research_2022} - Research paper & Research on slot allocation for airport network in the presence of uncertainty & Mixed integer programming, (Network of airports) & 15 coordinated airports in China (two calendar days) & The model minimizes the sum of displacement costs and delay costs under the worst-case scenario. \\
\midrule
17 & \citeonline{wang_slot_2023} - Research paper & Slot allocation for a multiple-airport system considering airspace capacity and flying time uncertainty & Chance constrained programming model (Network of airports)& Five airports within the Guangdong-Hong Kong-Macao Greater Bay Area (one calendar day) &  It outperforms those from the certainty model and the original schedule with the cost of a small number of increased slot displacements. \\
\midrule
18 & \citeonline{pellegrini_sosta_2017} - Research paper & SOSTA: An effective model for the simultaneous optimisation of airport slot allocation & Integer linear programming model (Network of airports) & European Level 2 and Level 3 airports (day with the highest air traffic volume of 2013) & It proved valuable in conducting a comprehensive sensitivity analysis of various model parameters and objective functions. \\
\midrule
19 & \citeonline{corolli_time_2014} - Research paper & The time slot allocation problem under uncertain capacity &  Two stochastic programming models (Network of airports) & Multiple European airport network datasets (4 calendar days) & The models can reduce schedule/request discrepancies and operational delays by up to 58\%. \\
\bottomrule
\end{tabular}
\end{sidewaystable}
\clearpage

\subsection{Exploring relationships within the data: Concept Mapping}

To address the information presented in each of the studies, by exploring their interrelationships through the topics of motivation, applied methods, obtained results, and limitations, thus illustrating the heterogeneity among them, the present systematic literature review developed a conceptual map.

The simplified map, illustrated in the \refFig{fig:narrative_two}, presents four subdivisions for enhanced comprehension: (i) Motivation for Improvement, wherein the principal reasons for the development of studies aimed at improving slot allocation at airports will be expounded and interrelated; (ii) Methods, wherein the approaches will be explained, along with their continuous developments and enhancements; (iii) Results, wherein, given the applied methods, the main contributions obtained are presented; and (iv) Limitations, wherein aspects not fully developed within the respective research, those indicated for future work, and gaps in the methodological approaches are elucidated.

The relationships among the studies within each subdivision will be elaborated upon in subsequent subsections. It is pertinent to note that the numerical designation of each study, as presented in the Table \ref{table*:SRL_narrative_table}, will be employed to simplify the connections between them.

\figuraBib{14a_narrative_section3_total}{Simplified conceptual mapping of the findings from Slot Allocation Problem \acrlong{SLR}}{}{fig:narrative_two}{width=0.4\textwidth}%

\subsubsection{Concept Mapping: Motivations for improvement}

Three principal branches originate from the motivations identified among the selected studies: (i) the anticipated increase in air traffic, attributed to the growing number of global passengers; (ii) the balance between efficiency and fairness in slot allocations, representing a pursuit of optimal infrastructure distribution that maximizes its utilization, alongside an approach ensuring that the scarce resource is allocated with transparency, non-discrimination, and equality; and (iii) the inherent complexity of the slot allocation problem, given its combinatorial nature. The \refFig{fig:narrative_three} illustrates the three main branches and their sub-branches, depicting motivations that are a result of the former.

The increasing development of the civil air transport industry is consistently identified across all studies as a primary reason to enhance the efficiency of slot management methods. This factor, combined with the current capacity constraints of airports – whether due to infrastructure limitations or regulations – leads to congestion. This congestion is responsible for widespread delays within airline networks, which can generate costs for all stakeholders in the industry \cite{ribeiro2018optimization, dixit_algorithmic_2023}.

Prior to the COVID-19 pandemic, global passenger volume was projected to reach 11.4 billion by 2024, representing 124\% of the 2019 level. Current projections for 2024 estimate a global passenger volume of 9.5 billion, which is 104\% of the 2019 level, indicating a 9\% year-over-year growth from the 2023 volume \cite{ACI_AirTravelDemand2025}. Consequently, despite the impact of the pandemic on the aviation industry, persistent congestion is anticipated at airports worldwide due to this continued growth in demand. This congestion often results in decreased flight punctuality and a diminished passenger experience \cite{liu_research_2022}.

\figuraBib{14b_narrative_section3_motivation}{Main motivations identified in the studies for addressing the airport slot allocation problem}{}{fig:narrative_three}{width=0.7\textwidth}%

The selected studies highlight the intensifying imbalance between rapidly growing air traffic and limited airspace, leading to reduced flight punctuality and increased delays \cite{corolli_time_2014, gillen2016airport, pellegrini_sosta_2017, jacquillat_interairline_2018, ribeiro2018optimization, jacquillat_roadmap_2018, zografos_bi-objective_2019, lambelho_assessing_2020, zeng_data-driven_2021, kuran_heuristic_nodate, keskin_optimal_2023, wang_slot_2023}. This situation underscores the critical need for effective slot allocation strategies to mitigate these adverse effects.

In 2019, the number of flights operated by passenger airlines in mainland China reached 4.611 million, a 6.1\% increase compared to the previous year \cite{wang_distribution_2022}. These authors, citing \citeonline{CAAC_StatisticalBulletin2019}, indicate a trend of increasing air passenger traffic in China. The average delay per passenger flight in 2019 was 14 minutes, representing a slight decrease compared to 2018. Such operational restrictions can lead to lost or displaced demand, negatively impacting the total revenue for stakeholders \cite{jacquillat_roadmap_2018}

\citeonline{benlic_heuristic_2018} shows a projection that indicates that air traffic in Europe will increase to 14.4 million flights by 2035. Consequently, in their most likely scenario, approximately 12\% of demand will be unaccommodated, leading to delays.

 \citeonline{corolli_time_2014} and \citeonline{zografos2017increasing} highlight that the rapid growth of air transport services in Europe, coupled with institutional limitations on the construction of new airports, has resulted in congestion issues manifested by increasing delays. In 2016, the average delay per flight within the European Civil Aviation Conference area was 18 minutes \cite{Eurocontrol_AnnualReport2016}. Furthermore, the authors note that \acrfull{ATM} inefficiencies in the European Union caused 10.8 million minutes of flight delays in 2012, incurring costs of €4.5 billion for airspace users and €6.7 billion for passengers, and generating 7.8 million tonnes of excess CO2 emissions \cite{IATA_SESFactsheet2014a}.

\citeonline{ribeiro2018optimization} cited data on flight punctuality in the United States and Europe. According to \citeonline{EUROCONTROL_USEuropeATM2024}, 80.1\% of flights arriving in the U.S. were within 15 minutes of their scheduled time, whereas in Europe, this figure was 76.5\%.

\citeonline{ribeiro_large-scale_2019} assert that the growth in air traffic demand has surpassed the available capacity at numerous airports globally, leading to the frequent occurrence of flight delays and substantial costs for airports, airlines, and passengers. To illustrate this assertion, the authors cite the financial impact of air traffic congestion in the United States, which exceeded \$30 billion in 2007 \cite{gillen2016airport, jacquillat_interairline_2018, jacquillat_roadmap_2018, ribeiro_large-scale_2019} and reached \$33 billion in 2019 \cite{dixit_algorithmic_2023}.

\citeonline{jacquillat_interairline_2018} and \cite{zografos_bi-objective_2019} demonstrated that a primary objective in addressing the slot allocation problem is to balance the preferences and requirements of diverse stakeholders. This often involves navigating trade-offs between efficiency (maximizing the aggregate utility of stakeholders), equity (fairly distributing utilities among stakeholders), and potentially other objectives such as maximizing outcome predictability and ensuring incentive compatibility. This perspective has led authors to propose models incorporating inter-airline fairness measures, which aim to distribute schedule displacement equitably among airlines \cite{fairbrother2018development, jacquillat_roadmap_2018,  zografos_bi-objective_2019, jiang_decision_2021}

These cited imbalances can lead to negative socio-economic and environmental impacts, including increased emissions, and a deterioration of the service quality offered. These consequences affect the travelling public, airlines, airports, and communities situated near airports \cite{keskin_optimal_2023}. Furthermore, \citeonline{zografos2017increasing} suggests that objectives could also be established to maximize social benefits.

Another key motivation, as highlighted by \citeonline{ribeiro_large-scale_2019}, is the limited applicability of exact optimization approaches, which primarily remains confined to small and medium-size airports. This limitation arises from the combinatorial complexity of the problem, rendering it computationally intractable for larger airports. Similarly, \citeonline{zografos2017increasing} supports this view, due to the inherent combinatorial complexity and resource constraints, it is unlikely that a solution algorithm with polynomial time complexity will be found. The complexity of the resulting slot scheduling problem has thus attracted significant interest from the research community, leading to the proposal of several models for optimizing slot allocation decisions \cite{jiang_decision_2021}.

\subsubsection{Concept Mapping: Methods and Results}

To address the research question formulated for this systematic literature review, as detailed in Section \ref{section:Planning}, it is necessary to examine the application of integer programming, queueing theory, and machine learning techniques in addressing the slot allocation problem, consistent with the conceptualization presented in \refFig{fig:narrative_one}. The selected studies have identified three primary methodological branches for solving the allocation problem: (i) single airport allocation and (ii) airport network allocation. A further approach involves (iii) reviewing the existing literature and synthesizing the methodologies and results, which constitutes the objective of the current \acrfull{SLR}. \refFig{fig:narrative_four} illustrates the methodologies employed in the reviewed studies alongside their main findings.

\figuraBib{14d_narrative_section3_method_results}{Methodological approaches and main findings of reviewed studies on slot allocation}{}{fig:narrative_four}{width=1.0\textwidth}%

\textit{Analysis of existing literature review on slot allocation problem}
\\

Initially, this report will address the secondary studies that provide a literature review.

\citeonline{zografos2017increasing} aimed to achieve three objectives: (i) a critical review of capacity declaration modeling and optimization strategies for slot allocation within available infrastructure; (ii) an identification of the challenges in existing research and its primary analytical gaps; and (iii) a proposal for future research directions to address the problem. The authors conducted a review of 96 studies, with around two-thirds cited in the article, primarily focusing on slot management approaches rather than market-based methods. The article highlights the necessity of approaches that lead to better quantity (e.g., more slots being accepted than through strict application of \acrshort{IATA} guidelines) and better quality (e.g., allocation closer to the initial submission request). It is demonstrated that a significant relationship exists between slot allocation, runway demand, aircraft fleet assignment, airport gate assignment, landing and takeoff sequencing, crew scheduling, as well as \acrshort{ATM} ground control or delay programming. \citeonline{zografos2017increasing} also addresses the modeling of airport infrastructure capacity declaration, showcasing various assumed attributes (e.g., level of aggregation, types of movements, time intervals for analysis, rolling constraints, capacity influenced by diverse internal and external factors, weather dependency, etc.). The authors not only explain the different types of approaches in terms of infrastructure (e.g., for a single airport, airport network) but also distribution methods (e.g., deterministic or heuristic). To this end, the formulation of an integer linear programming objective function is presented, within a deterministic allocation model, for both infrastructure categories – single and multiple airports. Finally, the article utilizes the unaddressed or explicitly stated needs for future study developments to generate a classification system for the development of new articles. This classification is based on geographical coverage, number of optimization criteria, type of objective function, nature of capacity, types of constraints, precedence relationships, priority classes, and other regulatory properties.

\citeonline{jacquillat_roadmap_2018} outline three primary objectives in their study: (i) the estimation of airport capacity and its key drivers; (ii) the presentation of an approach for determining punctuality performance as a function of demand and capacity; and (iii) the introduction of models that could potentially enhance airport demand management mechanisms, summarizing insights and providing a roadmap for future research. The authors define maximum throughput capacity as the “average number of aircraft movements that can be processed per unit of time under continuous demand,” identifying its main determinants as the number of runways, runway layout and configuration, separation requirements between movements, operational and meteorological conditions, arrival and departure mix, and aircraft mix. These factors directly influence an airport's ability to receive and accommodate air traffic demand. The authors highlight two ways to characterize airport capacity: theoretical, which utilizes the aforementioned data with certain simplifications to calculate the acceptable demand; and empirical, which, through historical data, estimates the Operational Throughput Envelope to characterize the maximum movements of each category that can be accommodated. Regarding airport congestion, the authors categorize models into three types: microscopic, which creates an airport layout and analyzes each aircraft individually, typically involving simulations; mesoscopic, which characterizes runway utilization, taxi-out and taxi-in times, along with other statistical data, to estimate demand; and macroscopic, which uses an aggregated representation of airport operations to generate accurate estimates of delays based on allocated flights and airport capacity. \citeonline{jacquillat_roadmap_2018} offer insights into the non-linearity between slots, airport capacity, and punctuality performance. They indicate that small increases in slot acceptance can lead to significant negative impacts on punctuality, whereas a more even distribution of flights throughout the day results in lower delay occurrences. The authors suggest that models such as queueing theory can be applied to understand these sensitivities and, consequently, be utilized with market mechanisms to alter the cost per extra movement during peak hours. In conclusion, the authors define a roadmap of actions following traffic forecast definition: airport infrastructure planning, ensuring the configuration is advantageous by avoiding both under-design (leading to lost demand) and over-design (leading to lost profitability); optimization of air traffic regulation and procedures, which implies training and technology application to meet demand safely, or even enable it; configuration of punctuality performance objectives, determining acceptable levels of congestion; and establishment of demand management mechanisms, whether based on administrative or economic principles.

\citeonline{gillen2016airport}, in their literature review, highlighted the interdependencies among the operating, managerial, and economic dimensions of the slot allocation problem. Their review aimed to identify models that (i) determine appropriate scheduling levels for congestion mitigation and (ii) specify mechanisms for allocating capacity to airlines. The article presented both administrative measures and market-based approaches, explaining and comparing mechanisms such as slot control, congestion pricing, and slot auctions. \citeonline{gillen2016airport} emphasized the inherent uncertainty in addressing delays, meteorological conditions, and the precise timing of movements, which are influenced by various stochastic factors. The authors also explained the complexity of air traffic operations, which can experience multiple bottlenecks across different phases (e.g., departure, en-route, and arrival), each with its own local constraints. Similar to \citeonline{jacquillat_roadmap_2018}, \citeonline{gillen2016airport} described models that assess demand performance status but do not focus on optimization. These include deterministic queue models, stochastic queue theory models, and their combinations. To address optimization, the authors cited Integer Programming, which aims to maximize the efficiency of slot allocation. Recognizing the complexity of choosing between price-based and quantity-based instruments, a decision dependent on several market characteristics, the authors established topics to guide the discussion on which method to apply. Finally, \citeonline{gillen2016airport} introduced an integrated framework for airport demand management, outlining the relationships between airport demand, supply, and on-time performance.

\hfill \break
\textit{Administrative mechanisms for airport slot allocation}
\\

Em construção


\subsubsection{Concept Mapping: Limitations}

Em construção

\figuraBib{14c_narrative_section3_limitation_2.drawio}{Theoretical understanding of the problem: Linking slot optimization and airport sensitivity to enhanced allocation and financial outcomes. Adapted from: \cite{popay2006guidance, eduspots_theoryofchange}}{}{fig:narrative_five}{width=0.7\textwidth}%

\subsection{Assessing the robustness of the synthesis: Critical Reflection}

Em construção