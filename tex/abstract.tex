Airport's slot allocation has impact for both airport revenue generation and airlines' level of service. When flights demand exceed the available airport's infrastructure capacity, and there is no opportunity to expand it, the slot coordination has to distribute the required flights thought a transparent approach, that efficiently uses the airport capacity without impacting the airlines' flights network. This paper aims to presents methods, focusing in models using integer programming (IP) and Machine Learning (ML), used to solve the airports' allocation problem for different perspectives.
\textit{Method:} Through a systematic literature review (SLR) of how this problem has been approached, the paper produced a review protocol, a research data and a report showing the knowledge assessed through multiple points of view, methods used, and the gaps for future research.
\textit{Results:} It was presented the main motivation for the problem approaches and methods used. Through a discussion it was showed an opportunity to include the ML methodology in order to obtain other perspectives, using past information.
\textit{Conclusion:} The SLR showed that there are several perspectives about the airports' slot allocation problem, such as for a small, medium and large airports, and for a single or multiple airport approach. For those airports, which the demand is not high, direct methodology have been applied, solving the optimization problem. For congested airports, the problem have been addressed through heuristics methods and there are yet gaps for computational optimization. The method (IP) is extensively applied in the problem, but there are yet opportunities, where the data is available, to apply ML methodology, using past information to extract data.