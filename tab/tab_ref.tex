\begin{center}
    
\setlength{\tabcolsep}{10pt} % Default value: 6pt
\renewcommand{\arraystretch}{1.5} % Default value: 1
\begin{xltabular}{\textwidth}{p{0.5cm} p{4cm} p{10cm}}
\caption{Glossary of relevant terms in WASG's slot management \cite{WASG2020}.} \label{tab:def} \\

\hline \multicolumn{1}{l}{\textbf{\#}} & \multicolumn{1}{l}{\textbf{Term}} & \multicolumn{1}{l}{\textbf{Definition}} \\ \hline 
\endfirsthead

\multicolumn{3}{c}%
{\tablename\ \thetable{} - Continued from previous page.} \\
\hline \multicolumn{1}{l}{\textbf{\#}} & \multicolumn{1}{l}{\textbf{Term}} & \multicolumn{1}{l}{\textbf{Definition}} \\ \hline 
\endhead

\hline \multicolumn{3}{r}{{Continued on next page.}} \\ \hline
\endfoot

\hline
\endlastfoot

1 & Slots & A permission that is given by a coordinator for a planned operation to use an airport's infrastructure at a specific date and time. \\ 
2 & Airport Level & There are three airport levels: (1) for those whose demand does not exceed the infrastructure capacity; (2) for those whose demand causes certain congestion at specific times (e.g., hours, days, days of week), but which can be resolved through mutual agreements between the facilitator and air operators; and (3) for those whose demand exceeds the airport capacity limits, requiring a coordinator for the slots allocation, using WASG's best practices and prioritizing allocations for better market competitiveness. \\
3 & Infrastructure capacity and airport parameters & Compiled of the necessary parameters for the coordination of slots. Through these, it is possible to identify the operational capacity for allocation that does not exceed the demand limit, providing an adequate service level (e.g., maximum operations per hour / 30 min. / 15 min. / 5 min. on the runway, number of aircraft parking positions available on the apron, number of available gates, airspace limitations, environmental limitations, etc.). \\
4 & Seasons & There are two specific periods where operations take place. The (1) Summer Season, which starts on the last Sunday of March, and the Winter Season, which starts on the last Sunday in October. \\
5 & Series of slots & A minimum of 5 slots allocated for approximately the same time on the same day of the week through a season. \\
6 & Slot pool & All slots that will receive allocation priority at level 3 airports, after the historical slots are properly allocated. \\
7 & Historic Slots & Slots with operating precedence at the allocated airport, acquired by a regularity above 80\% in the equivalent previous season. \\
8 & Facilitator & The one responsible for collecting data and adjusting movement at level 2 airports. \\
9 & Coordinator & The one responsible for data collection and coordination of slots at level 3 airports. \\
10 & Activity Calendar & Deadlines and Events that manage the process of coordinating movement and slots for each season. It is established two per year. \\
11 & Previous Equivalent Season & Last Season of the same name, i.e., if the current season is a summer season, the previous equivalent season is the prior summer season. \\
12 & New entrants & Air operators that have a small number of operations (less than 7) for each day of a season, or companies that do not operate yet and are requesting operations. \\
13 & Annual movements & Are all movements that have specific times in both Summer and Winter seasons. \\
14 & Messages & Reference to data shared between coordinator, facilitator and air operators \cite{SSIM2020}. The messages are usually standardized in a format recognized by the industry (e.g., SSIM), containing critical information about the slot, i.e., operations' start date, operations' end date, air operator IATA's code, equipment type (e.g., Boeing 737 MAX), number of seats offered, type of movement air (e.g., Passenger or Cargo), frequency of operation during the week, hours of operation (slots), among others.\\ \hline
\end{xltabular}

\end{center}