\clearpage

\begin{sidewaystable}

\caption{Primary studies collection} \label{table*:SRL_titles}
\tiny
%\begin{tabular}{llllll}
\begin{tabular}{|p{0.25cm} p{2.5cm} p{5cm} p{15.5cm}|}
\toprule
\centering

\textbf{\#} & \textbf{CITATION} & \textbf{TITLE} & \textbf{FINDINGS} \\

\midrule

\textbf{1}  & \citeonline{zeng_data-driven_2021} - Research paper & A data-driven flight schedule optimization model considering the uncertainty of operational displacement & An optimization model is developed to minimize deviations between scheduled and actual flight operations, incorporating airspace corridor flow as a constraint. By leveraging historical data, this model aims to reduce overall airport delays. Using Hangzhou \acrfull{HGH} as a case study, this research demonstrates a reduction of over 60\% in both arrival and departure traffic compared to other optimization models. However, while improvements in delay reduction are observed, the model exhibits higher flight adjustment rates and a greater maximum displacement per flight, highlighting the need for further investigation into the trade-offs between delay and displacement criteria.\\
\midrule
\textbf{2}  & \citeonline{ribeiro_large-scale_2019} - Research paper & A large-scale neighborhood search approach to airport slot allocation &  This article introduces a \acrfull{PSAM-RTA}. Building upon the existing \acrshort{PSAM} \cite{ribeiro2018optimization}, the new model integrates terminal and apron capacity limitations. Notably, apron constraints create dependencies across different time periods, substantially increasing the model's complexity. To address this, the authors propose a two-stage approach: first, a constructive heuristic rapidly generates a good initial feasible solution by prioritizing slot requests based on their frequency; second, an improvement heuristic refines this solution by iteratively breaking down the problem and re-optimizing slot allocations within defined neighborhoods. The \acrshort{PSAM-RTA} model was then applied to \acrfull{LIS}, which handles over 200,000 annual flight movements. The authors' algorithm produced solutions within 0.1\% of the optimum in just a few hours. In contrast, standard commercial optimization solvers struggled to find the optimal solution, exhibiting a 2\%–5\% optimality gap even after seven days of computation, demonstrating a significant improvement.\\
\midrule
\textbf{3}  &  \citeonline{zografos2017increasing} - Integrated review & Increasing airport capacity utilisation through optimum slot scheduling: review of current developments and identification of future needs &  This paper critically reviews current research in declared capacity modeling and strategic slot scheduling. It identifies future research issues and gaps, and proposes concrete directions for modeling and solving advanced slot scheduling problems at both single airports and across networks. Research findings indicate that the next generation of slot scheduling models should explore variation. Overall, this review encompassed 96 research papers published between 1993 and 2014, with nearly two-thirds cited herein. Beyond producing and explaining slot scheduling problem models, this paper also develops a classification system for the approaches used. This system is based on geographical coverage, number of optimization criteria, objective function, capacity nature, resource constraints, precedence relationships, priority classes, and other regulatory properties.\\
\midrule
\textbf{4}  &  \citeonline{ribeiro2018optimization} - Research paper & An optimization approach for airport slot allocation under \acrshort{IATA} guidelines & This paper proposes a novel \acrfull{PSAM} to optimize slot allocation decisions considering both slot availability and airline slot requests. The model minimizes schedule coordination costs for airlines, quantified by the deviation from airline requests, while adhering to the various priorities and requirements outlined in the \acrshort{IATA} guidelines.   
The case studies presented in this paper utilize slot request and allocation data from the Summer Season of 2014 at Madeira and Porto airports. Comparisons with slot coordinator decisions indicate that the model accurately reflects the key decisions and trade-offs made in practice. Furthermore, the model improves slot allocation outcomes by reducing the displacement experienced by airlines by an estimated 4.5\% at Madeira and 27\% at Porto.
However, the paper acknowledges certain limitations: it does not fully incorporate additional complexities such as terminal, apron, and noise restrictions; the original formulation needs strengthening to enhance computational times; it does not fully explore the trade-offs or alternative methods to better balance connection integrity with slot allocation efficiency; and certain assumptions, like treating all flights within a series identically, may not capture all real-world nuances and potential imbalances across different days or flight types.
\\
\midrule
\textbf{5}  &  \citeonline{benlic_heuristic_2018} - Research paper & Heuristic search for allocation of slots at network level & This paper addresses the strategic allocation of slots to airlines for an entire scheduling season, while considering slot complementarity and pairing issues at a network level.
It extends the single-airport slot management model developed by \citeonline{zografos2012dealing}, which incorporates key IATA rules: (i) airport capacity limitations, (ii) request priorities, (iii) allocation of slot series rather than individual slots, and (iv) minimum turnaround time between an aircraft's arrival and departure. Beyond including the series of slot" constraint.  The presented model differs from the recent model by \citeonline{pellegrini_sosta_2017} through the incorporation of en-route constraints.
The paper proposes a two-stage heuristic approach. First, a constructive heuristic procedure generates a feasible initial solution by eliminating rotations (sequences of legs assigned to an aircraft) for which a coherent allocation of slots at origin and destination airports cannot be found. Second, an iterative heuristic optimizes this initial solution by minimizing schedule delay (the time difference between requested and allocated slots).
To evaluate the model, the authors generated a set of benchmark instances varying in the number of airports (up to 100) and the daily distribution of movements (up to a total of 4,620,854 aircraft movements considered on a half-yearly basis). However, the instance creation involved simplifying assumptions such as constant daily airport capacity, uniform traffic distributions, and scaling methods that may not fully reflect the variations observed in real-world airports.
The quality of a schedule is evaluated through comparisons with schedules obtained when the en-route constraint is disregarded, effectively treating each airport in the network independently.
The proposed approach demonstrates that even under strict rules (such as priority classes and slot series), a network-wide allocation method can achieve a coherent schedule that minimizes delays. This is particularly beneficial in congested networks where traditional methods might lead to a higher number of unmatched or unaccommodated requests.
Furthermore, the paper explores the possibility of relaxing certain constraints (like slot series) in highly constrained scenarios, potentially increasing the number of accommodated operations without significantly compromising schedule quality.
\\
\midrule
\textbf{6}  & \citeonline{wang_distribution_2022} - Research paper & Distribution prediction of strategic flight delays via machine learning methods & This article employs three machine learning algorithms — \acrfull{MLP}, \acrfull{LGBM}, and \acrfull{RF} — to predict the distribution of flight delays.
The algorithms were applied to the scheduled flight data of \acrfull{CAN}, encompassing six strategic flight schedules that cover all flights operated from March 26, 2017, to March 28, 2020.
To optimize the performance of each algorithm, a random search algorithm was utilized to identify the optimal hyperparameters.
The results indicate that the prediction of departure delay distribution is significantly more accurate than that of arrival delays. Prediction accuracy exceeding 0.80 was achieved for departure delays at a 0.65 confidence level and for arrival delays at a 0.50 confidence level. The \acrshort{MLP} algorithm using a quantile loss function, \acrshort{LGBM}, and \acrshort{RF} algorithms exhibited comparable performance. Specifically, the \acrshort{MLP} (quantile) algorithm demonstrated the best performance in predicting departure delay distributions.
The article identifies several limitations. Given the numerous factors influencing flight delays, fitting the delay distribution for individual flights with multiple normal distributions could be beneficial. Furthermore, the study does not predict cancellations due to data limitations. Finally, the prediction performance could potentially be improved by developing a more sophisticated model or by employing a more precise loss function tailored to each machine learning algorithm.
\\
\midrule
\textbf{7}  &  \citeonline{jacquillat_interairline_2018} - Research paper & Interairline equity in airport scheduling interventions & The approach developed in this paper provides a methodological foundation to address airport capacity allocation problems aimed at mitigating delay externalities, promoting airline competition, and maximizing social welfare while ensuring inter-airline equity.

This paper designs, optimizes, and assesses a novel approach for airport scheduling interventions that incorporates inter-airline equity objectives.

It relies on a lexicographic modeling architecture based on efficiency, equity, and on-time performance objectives, subject to scheduling and network connectivity constraints. The paper develops quantitative indicators for each objective within a unified framework of scheduling interventions, and formulates a tractable lexicographic architecture to characterize and optimize the trade-offs between these objectives in airport scheduling interventions.

The proposed approach is used to solve real-world, full-scale computational scenarios at John F. Kennedy Airport (JFK).

The results show that achieving maximum equity requires no (or minimal) sacrifice in terms of efficiency losses. Conversely, for some computational scenarios, the findings indicate that neglecting inter-airline equity (i.e., considering efficiency-based objectives exclusively, or, in some cases, requiring maximum efficiency) can lead to highly inequitable outcomes.

Most importantly, this paper assumes knowledge of the scheduling inputs provided by the airlines. Therefore, future research could analyze the strategic interactions among airlines and minimize the potential for manipulation.\\
\midrule
\textbf{8}  &  \citeonline{zografos_bi-objective_2019}  & A bi-objective efficiency-fairness model for scheduling slots at congested airports & \\
\midrule
\textbf{9}  &  \citeonline{jiang_decision_2021} & A decision making framework for incorporating fairness in allocating slots at capacity-constrained airports & \\
\midrule
\textbf{10} &  \citeonline{jacquillat_roadmap_2018} - Integrated review & A roadmap toward airport demand and capacity management & \\
\midrule
\textbf{11} &  \citeonline{gillen2016airport} - Integrated review & Airport demand management: The operations research and economics perspectives and potential synergies  &  This research provides a framework that underscores the critical interdependencies between operational/managerial, and economic considerations in airport demand management. This framework aims to facilitate the development of more effective demand management policies at busy airports worldwide. The study highlights a significant divergence in scope and approach between the Operations Research/Management Science (OR/MS) and economics literatures, despite their shared objective of managing airport demand. While OR/MS research primarily focuses on estimating airport capacities and developing models for optimizing operational dynamics, economic research tends to concentrate on analyzing airline scheduling incentives and examining the impact of market structures on airport congestion.\\
\midrule
\textbf{12} &   \citeonline{dixit_algorithmic_2023} - Research paper & Algorithmic mechanism design for egalitarian and congestion-aware airport slot allocation & A game-theoretic model and a mechanism design solution, specifically the Egalitarian and Congestion-Aware Truthful Slot (ECATS) mechanism, to ensure fair and efficient slot allocation at congested airports. To evaluate its effectiveness, the model was applied to Indira Gandhi International Airport (DEL) and Chennai International Airport (MAA) using real-world flight data from five representative days. Compared to the IATA-based state-of-the-art allocation methods and current practices, the ECATS mechanism yielded a 5-20\% and 20-30\% increase in social utility at DEL and MAA, respectively.\\
\midrule
\textbf{13} &  \citeonline{kuran_heuristic_nodate} - Thesis & Heuristic optimization methods for seasonal airport slot allocation & This research aims to maximize the number of confirmed requests within the airport slot allocation process at Vienna Airport. To achieve this, the study develops a novel approach that approximates the Pareto frontier of the multi-objective problem across five distinct scheduling seasons. The proposed method demonstrates promising results in minimizing time deviations from the requested slots. However, its performance is observed to be sensitive to the degree of slot series fragmentation. It explores the Pareto frontier between slot deviations and fragmentation to effectively balance these competing objectives. Furthermore, the analysis reveals that the proposed method outperformed the referential data in four out of the five analyzed seasons for runway capacity. It can also be applied for passenger restrictions airports.\\
\midrule
\textbf{14} & \citeonline{lambelho_assessing_2020}  - Research paper & Assessing strategic flight schedules at an airport using  machine learning-based flight delay and cancellation predictions  & This research investigates the application of machine learning algorithms, specifically LightGBM, \acrfull{MLP}, and \acrfull{RF}, for predicting flight delays and cancellations at \acrfull{LHR} up to six months in advance. It uses 10 seasons as dataset. Furthermore, this research proposes ranking 10 \acrfull{KPI} associated with predicting flight cancellations and delays within the context of strategic scheduling. The proposed prediction models demonstrated high accuracy (0.79) across all evaluated approaches. LightGBM consistently exhibited superior performance. \\
\midrule
\textbf{15} & \citeonline{keskin_optimal_2023} - Research paper & Optimal network-wide adjustments of initial airport slot allocations with connectivity and fairness objectives & This research introduces a novel approach to airport slot allocation by developing bi-objective mathematical models that simultaneously consider both schedule efficiency and inter-airline fairness. These models incorporate the crucial role of individual airports within the broader air transport network by integrating IATA connectivity indices and betweenness centrality measures. The model was applied to a network of 16 coordinated and facilitated airports in Brazil during the 2019 summer scheduling season. The case study results indicate that using connectivity metrics significantly affects the distribution of total displacement among airports, although the network-wide total displacement itself is not significantly impacted. The analysis suggests that a fairer network-wide schedule can be achieved without sacrificing efficiency, as expressed through the minimization of total displacement.  Airports with the highest centrality and connectivity values receive proportionally less total displacement relative to the number of flights they operate.\\
\midrule
\textbf{16} & \citeonline{liu_research_2022} - Research paper  & Research on slot allocation for airport network in the presence of uncertainty   &It introduces a two-stage stochastic programming model to optimize airport slot allocation within a multi-airport system. The model minimizes the sum of displacement costs and delay costs under the worst-case scenario, leveraging a Benders Decomposition-based algorithm for an efficient solution. The model was applied to a network of 15 coordinated airports in China, considering two operational days and five different scenarios. \\
\midrule
\textbf{17} &   \citeonline{wang_slot_2023} - Research paper & Slot allocation for a multiple-airport system considering airspace capacity and flying time uncertainty &This research introduces a novel chance-constrained slot allocation model for a multi-airport system (MAS) that optimizes slot assignments while explicitly considering the inherent uncertainty associated with airport capacity constraints. The model was applied to a case study encompassing five airports within the Guangdong-Hong Kong-Macao Greater Bay Area, utilizing actual flight schedules as input. It outperforms those from the certainty model and the original schedule with the cost of a small number of increased slot displacements.\\
\midrule
\textbf{18} &  \citeonline{pellegrini_sosta_2017} - Research paper & SOSTA: An effective model for the simultaneous optimisation of airport slot allocation &  Simultaneous Optimization of the airport Slot Allocation (SOSTA) is an integer linear programming model designed to optimize the airport slot allocation process. The model was applied to all Level 2 and Level 3 airports across Europe, utilizing real-world flight data from the highest air traffic volume of 2013. SOSTA could not provide a large improvement with respect to the allocation. However, the SOSTA model proved valuable in conducting a comprehensive sensitivity analysis of various model parameters and objective functions, demonstrating robust computational performance \\
\midrule
\textbf{19} &  \citeonline{corolli_time_2014} - Research paper & The time slot allocation problem under uncertain capacity & This model addresses the dynamic nature of airport slot allocation by incorporating uncertain capacity availability, based on a two-stage stochastic programming framework. The model was evaluated on a network of 21 European airports, utilizing four distinct datasets in four calendar days. In one test instance, the time-linked recourse formulation demonstrated a significant reduction in total delay costs, exceeding 58\%  \\
\bottomrule
\end{tabular}
\end{sidewaystable}